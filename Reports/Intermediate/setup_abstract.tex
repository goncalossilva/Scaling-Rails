\chapter*{Abstract}
\pdfbookmark[0]{Abstract}{abstract}

Web's popularity and importance on everyday life increases day by day. Its users create new standards and expectations, demanding better user experiences in their daily interactions. Availability and response times are key factors to every website's success. The noticeable Web growth encouraged the creation of better tools to improve the quality of web applications. One of this tools is Ruby on Rails, a framework built on top of Ruby's programming language.

Rails scalability is questionable since many platforms experienced performance and availability difficulties when having to respond to increased popularity. One of these platforms was Twitter. 

Rails performance is dependent on all the underlying components like the operating system, ruby, web server, database, rails itself and its superjacent application. A key concept in Rails' performance tuning and optimization is to analyze and improve each component from the framework's point of view. No component should be optimized as an individual, independent part --- the whole system must be taken into account when optimizing, considering each component's sensitive dependencies.

This report is a review on the state of the art in scaling Rails. It starts by exposing reliable software alternatives to each component and explains their characteristics, principles, philosophies and architecture, providing a solid base for their analysis. Their state of the art is reviewed, exhibiting related work and its results. A problem-solving approach to this issue is then proposed, along with its respective scheduling which will be endured at \textit{Escolinhas}, an expanding Ruby on Rails application. Finally, an analysis is performed on the aforementioned related work and future options are narrowed down according to other research's results.

\chapter*{Resumo}
\pdfbookmark[0]{Resumo}{resumo}

A popularidade e importância da Web na vida quotidiana aumenta de dia para dia. Os seus utilizadores criam novos padrões e expectativas, exigindo melhores experiências de utilização na sua interacção com a plataforma. A disponibilidade e tempo de resposta de um sítio na Internet revelam-se como sendo factores críticos de sucesso. O notório crescimento da Web motivou a criação de melhores ferramentas que permitissem desenvolver aplicações com qualidade superior. Uma destas ferramentas é o Ruby on Rails, escrita na linguagem de programação Ruby.

A escalabilidade do Rails é questionável já que várias plataformas apresentaram problemas de eficiência e disponibilidade quando viram a sua popularidade aumentada. Uma destas plataformas foi o Twitter.

O desempenho do Rails depende de todos os seus componentes subjacentes como o sistema operativo, o ruby, o servidor web, a base de dados, o próprio Rails e a aplicação sobrejacente. Um conceito fundamental na optimização de desempenho do Rails consiste em analisar e melhorar cada componente do ponto de vista do Rails. Nenhum componente deve ser optimizado como se fosse uma parte independente do sistema --- todos os componentes devem ser tidos em conta quando se optimiza, tendo sempre em conta o impacto que estes poderão ter nos restantes.

Neste relatório é apresentada uma revisão do estado da arte sobre a escalabilidade do Rails. Começa-se por expor alternativas viáveis para cada componente e explica-se as suas características, pormenores e arquitectura, criando uma base sólida para a sua análise. O estado da arte de cada uma destas alternativas é então revisto, expondo trabalhos relacionados e as suas conclusões. Uma aproximação à solução deste problema é proposta, acompanhada da sua calendarização, que será aplicada no contexto do \textit{Escolinhas}, uma aplicação em crescimento desenvolvida em Ruby on Rails. Por fim, é feita uma análise sobre as alternativas supracitadas e estas são limitadas pelas descobertas feitas em projectos realizados.
