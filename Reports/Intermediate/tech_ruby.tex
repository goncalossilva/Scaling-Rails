\section{Ruby} % (fold)
\label{tech:sec:ruby}
Ruby is a dynamic and object-oriented language created by Yukihiro Matsumoto, who released it to the public in 1995. The purpose was to create a ``language that was more powerful than Perl, and more object-oriented than Python''~\cite{interview_creator_ruby}.

Ruby was inspired by languages such as Lisp, Smalltalk and Perl, and its core characteristics and features are~\cite{ruby_about, ruby_book}:
\begin{description}
  \item[Open source.] Ruby's license allows anyone to use, copy, modify or distribute it.
  \item[Pure object-oriented.] In Ruby, \emph{everything} is an object, including classes, modules and data types --- even numbers, booleans and null values which in other object-oriented languages are known as \emph{primitives}. An object's properties are known as \emph{instance variables} and actions are \emph{methods}.
  \item[Flexible.] It not only features dynamic typing, but also very powerful reflective and metaprogramming capabilities. Ruby doesn't restrict what a programmer can do. Any part of Ruby code can be removed, redefined or extended, even at \emph{runtime}. This isn't only true for a programmer's own code, but also for core Ruby classes such as Object and String.
 \item[Automatic memory management.] Like other languages such as Java and contrary to C, developers don't manage the program's memory usage.
  \item[Portable.] Ruby is mainly developed in GNU/Linux but can run on most operating systems and platforms, such BSD, Mac OS X, Windows 95 and many others.
  \item[Exception handling.] Ruby can recover from errors just like Java, C++ or Python.
\end{description}
Ruby started to become popular in 2001, with the start of the~\textit{RubyGems} project, which enables packaging and distribution of applications and libraries in an easy way~\cite{ railssolutions}. Ruby has two main versions and the last one was recently released~\cite{ ruby191_release}.


\subsection{Ruby 1.8}
The most commonly found version of Ruby in Rails' projects is 1.8, mainly due to the fact that it's the officially recommended version. ~\cite{rubyonrails_download}.

\subsubsection{MRI}
Ruby's 1.8 official interpreter was developed by the language's creator Yukihiro Matsumoto, also known as \emph{Matz} and its first release happened in 2003. Some people mistake MRI for ``Main Ruby Implementation'' but the abbreviation is actually related to its creator's name --- \emph{Matz Ruby Interpreter}. Some particular characteristics of this interpreter include:
\begin{description}
\item[Language.] MRI is written in C.
\item[Threading.] It can run multiple threads without relying on the operating system capabilities by using \textit{green threads}.
\item[Garbage Collector.] The garbage collector is based on the simple \textit{mark-and-sweep} algorithm.
\item[Extensions.]  Developers can extend Ruby's basic functionality by writing their own extensions. These can be written in Ruby or can be native C extensions, by using MRI's powerful API.
\item[Bytecode Interpretation.] MRI lacks a bytecode interpreter. When a program is executed, MRI will parse its source and create its syntax tree. Then, it will iterate over this tree directly while executing the program.
\end{description}
Since it's the official Ruby interpreter, it's the most commonly used. 


\subsubsection{Ruby Enterprise Edition}
This Ruby interpreter, also called REE, is based on the MRI source code. However, it includes many Rails-oriented enhancements. It was first released in 1995 and has been merging with MRI periodically, keeping its own changes aside. Its characteristics and differences from the official interpreter – MRI - include~\cite{rubyenterpriseedition}:
\begin{description}
\item[Language.] Since REE is based on the MRI, it's written in C.
\item[Threading.] The thread implementation is the same found on MRI.
\item[Garbage Collector.]  The garbage collector was improved, being \textit{copy-on-write} friendly. This allows for reduced memory usage when paired with Passenger, who uses \textit{preforks} in combination with this feature.  It also enables the user to tweak its settings for adaptive performance.
\item[Extensions.]  Just like MRI, it supports Ruby and native C extensions. 
\item[Other Differences.] REE uses \textit{tcmalloc} for memory allocation, which improves the process's performance. It also allows better debugging, by introducing the ability to inspect the garbage collectors' state and also to dump stack traces for all running threads.
\item[Bytecode Interpretation.] REE is mostly based in MRI's source code, as mentioned before, so it lacks a bytecode interpreter. It also iterates over the program's syntax tree directly.
 \end{description}
This Ruby interpreter is normally paired with Phusion's web server --- Passenger. Since they're developed by the same team, some optimizations are more noticeable when these two are being used together.


\subsubsection{JRuby}
JRuby is a Java implementation of Ruby whose first version to support Rails was released in 2006. It has many differences from the MRI and these include~\cite{mri_jruby_comparison, jruby_performance_glassfish}:
\begin{description}
\item[Language.] JRuby is written in Java, running on top of JVM.
\item[Threading.] The thread implementation is based on JVM threads. These are more efficient than the \textit{green threads} used by MRI.
\item[Garbage Collector.]  The garbage collector is inherited from Java, being generational based. It also inherits its heap and memory management, granting it greater performance since Java's memory management is overall very efficient.
\item[Extensions.]  JRuby does not support native C extensions, but the most popular ones have already been ported to Java.
\item[Other Differences.] JRuby doesn't support continuations and forks. There are also a few quirks around its native endians and time precision. It does, however, support the Ruby 1.8 specification to full extent and is currently used in production.
\item[Bytecode Interpretation.] In this concern, JRuby has the advantage of running on top of the JVM. The Ruby code can be interpreted directly, like MRI's behavior, but it can also be targeted for \textit{just-in-time} or \textit{ahead-of-time} compilation to Java bytecode, which is handled very efficiently by the JVM.
\end{description}
This Ruby interpreter's development has involved companies like Sun Microsystems~\cite{sun_jruby}, ThoughtWorks~\cite{thoughtworks_jruby} and Engine Yard~\cite{engineyard_jruby}.


\subsubsection{Rubinius}
Rubinius is another \textit{work-in-progress} implementation of the Ruby programming language. It has a great deal of focus on performance and its characteristics include~\cite{rubinius, rubinius_virtual_machine, rubinius_virtual_machine_rewrite}:
\begin{description}
\item[Language.] Rubinius implements the Ruby programming language in C++. Although being initially developed in C, it was soon ported to C++ since it allowed many enhancements like stronger typing, usage of STL types, better match between the VM and Ruby semantics and others.
\item[Threading.] Rubinius supports native threads, allowing it to execute in different processors if the operating system allows it.
\item[Garbage Collector.]  The garbage collector is based on a generational algorithm, similar to Java's. Nonetheless, it is implemented in C++ making it a bit more efficient.
\item[Extensions.]  Rubinius supports Ruby and native C extensions by providing a compatible C-API for extensions written for the standard Ruby interpreter.
\item[Bytecode Interpretation.] Rubinius uses a bytecode compiler written entirely in Ruby. It then uses LLVM to compile bytecode to machine code at runtime, providing very efficient program executions.
\end{description}
Unfortunately Rubinius only supports 93\% of the Ruby specification, according to the pass rate test from RubySpec. This interpreter is, however, a very promising Ruby interpreter and its compatibility with the language specification is growing each day.


\subsection{Ruby 1.9}
This version represents a step forward in the Ruby programming language. It includes many new features and enhancements. Some of these include \textit{Fibers} and \textit{Non-blocking I/O} improvements~\cite{changes_ruby19}.


\subsubsection{YARV}
Yet Another RubyVM, also known as YARV, has been adopted as the official MRI's successor as the interpreter for version 1.9. For this same reason some people call it KRI or \textit{Koichi's Ruby Interpreter}. It consists on a bytecode interpreter designed specifically for Ruby and it's the only to fully support version 1.9's specification. Its author purpose, on its creation, was to reduce execution times of Ruby programs and it differs a bit from other implementations~\cite{yarv, rubyvm_interview, ruby_intermediate_language}:
\begin{description}
\item[Language.] YARV was developed in C and reuses many parts of its predecessor, like the Ruby script parser, the object management mechanisms and the garbage collector.
\item[Threading.] Contrary to MRI, YARV supports native threads. It also efficiently supports Fibers, previously called continuations, without suffering of serious performance issues like its predecessor~\cite{memory_leak_fix_18X}.
\item[Garbage Collector.]  As aforementioned, YARV reuses MRI's code for its garbage collector.
\item[Extensions.]  YARV supports its predecessor's extensions either they are written in Ruby or C. This represents, however, a bottleneck on parallel computing because of existing extensions' synchronization issues.
\item[Bytecode Interpretation.] As mentioned before, the main difference between MRI and YARV when it comes to performance is related to the last's generation of intermediate code, which is much faster to process than parsing the program's syntax tree nodes one by one.
\item[Other Differences.] There are many differences between YARV and MRI, our reference interpreter. An important one is the usage of Fibers which allows the developer to do cooperative scheduling instead of using the preemptive context switch model, commonly used in threading scheduling. On the same subject, Fibers are cheaper to create than threads. 
\end{description}
YARV had an initial great impact in the Ruby community for its enhancements. It brings many advantages as new features, better performance and improved memory usage. However, it has not been extensively adopted because of existent incompatibilities with some libraries which have not been upgraded to comply with Ruby's new specification~\cite{rubys_challenge_2009}.
