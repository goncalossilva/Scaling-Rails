\section{Databases} % (fold)
\label{state:sec:databases}
The database is responsible for persisting data. It stores objects information so they can be retrieved latter without needing to use the system's memory.
The database usually represents a worrying bottleneck for high performance applications. It interacts with the disk --- the system's main memory --- and in recent hardware this represents a performance bottleneck~\cite{memory_wall}. Accessing the disk is a costly procedure and web applications need a very efficient database in order to diminish this bottleneck's impact.

Across most relational databases, from MySQL to PostgreSQL and including proprietary DBMS, MySQL consistently outperforms all others~\cite{benchmark_relational_databases}.
Despite being the fastest schema-based database in production, a schema-less database --- MongoDB --- improves over MySQL's performance significantly. This kind of database is more flexible but it also uses more disk space, as row names are stored along with their content to enable this flexibility. MongoDB is not atomic, however, not enabling useful features like transactions ~\cite{mysql_to_mongodb}. This is not related to the fact of being a schema-less database but with the databases' philosophy, since CouchDB --- another well-known schema-free database created by the Apache Foundation --- supports transactions on a performance penalty.

A critical procedure to improve a system's scalability is database caching. The reduction of disk accesses provides huge performance improvements~\cite{scaling_rails_bottomup} allowing improvements grater than 100\%~\cite{rapid_prototyping_mdd,high_performance_database_caching}.

