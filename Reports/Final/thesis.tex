% FEUP THESIS STYLE for LaTeX2e
% how to use feupteses (changed from the original for MIEEC)
%
% FEUP, JCL & JCF, Tue May 20 18:53:15 2008
%
% PLEASE send improvements to jlopes at fe.up.pt, jcf at fe.up.pt
%

%%========================================
%% Commands: pdflatex mieic
%%           bibtex mieic
%%           makeindex mieic (only if crating an index) 
%%           pdflatex mieic
%%========================================

%% For one side layout comment next line and uncomment the second line
\documentclass[11pt,a4paper,twoside,openright]{report}
\usepackage[utf8]{inputenx}
\usepackage[official]{eurosym}
\usepackage[english]{babel}
\usepackage{amsmath}
\usepackage{amsfonts}
\usepackage{tkz-graph}
\usepackage{multirow}
\usepackage{longtable} % needs font/line height hack (it's local right now)

\usepackage{verbatim}
\makeatletter 
\g@addto@macro\@verbatim\small 
\makeatother 

\usepackage{color}
\definecolor{grey}{rgb}{0.53,0.53,0.53}
\definecolor{red}{rgb}{0.87,0.13,0.0}
\definecolor{green}{rgb}{0.0,0.53,0.0}

\usepackage{listings}
\lstset{
language=Ruby,                   % choose the language of the code
basicstyle=\footnotesize,       % the size of the fonts that are used for the code
backgroundcolor=\color{white},  % choose the background color. You must add \usepackage{color}
showspaces=false,               % show spaces adding particular underscores
showstringspaces=false,         % underline spaces within strings
showtabs=false,                 % show tabs within strings adding particular underscores
frame=single,	                  % adds a frame around the code
tabsize=2,	                    % sets default tabsize to 4 spaces
captionpos=t,                   % sets the caption-position to bottom
breaklines=true,                % sets automatic line breaking
numbers=left,                   % where to put the line-numbers
numberstyle=\scriptsize,        % the size of the fonts that are used for the line-numbers
stepnumber=1,                   % the step between two line-numbers. If it's 1 each line will be numbered
numbersep=5pt,                  % how far the line-numbers are from the code
keywordstyle=\bf \color{red},
stringstyle=\color{green},
identifierstyle=\color{black},
commentstyle=\color{grey},
%numberstyle=\color{darkgreen},
}
\usetikzlibrary{calc,arrows,shapes}

\usepackage{enumerate}

%% For the final version, comment next line and uncomment the second line
%\usepackage[provisional,alpharefs]{feupteses}
\usepackage[alpharefs,print]{feupteses} 

%% Options: 
%% - portuges: titles, etc in portuguese
%% - provisional: the thesis has not been approved yet
%% - alpharefs: bibliography references are alphabetic
%% - numericrefs: bibliography references are numbered (in order of citation)
%% ( by default: author-date format of the ``natbib'' package is used 
%%   the portuguese version requires the file ``plainnat-pt.bst'' to be 
%%   present in the same directory )

%% Include MIEIC definitions different from standard style
\usepackage{mieicpatch}

\version{1.0}

%% Uncomment in the final version in order to make version footer disappear
\noversiontrue
%% Uncomment to create an index (at the end of the document)
\usepackage{makeidx}
\makeindex
%% Path to the figures directory
%% TIP: use folder ``figures'' to keep all your figures
\graphicspath{{figures/}}

%%========================================
%% Macros
% format
\newcommand{\class}[1]{{\normalfont\slshape #1\/}}

% entities
\newcommand{\Feup}{Faculdade de Engenharia da Universidade do Porto}
\newcommand{\ProjectAuthor}{Gonçalo Santarém da Silva}
\newcommand{\ProjectSupervisor}{Ademar Aguiar}

% names
\newcommand{\ProjectName}{Scaling Rails: a system-wide approach to performance optimization}}
\newcommand{\headermark}[1]{\chaptermark{\uppercase{#1}}}

% Gantt
\newcommand\ganttline[4]{% line, tag, start end
\node at (0,#1) [anchor=base east] {#2};
\fill[gray] (#3/15,#1-.2) rectangle (#4/15,#1+.2);
\draw[black,thick] (#3/15,#1-.2) rectangle (#4/15,#1+.2);}

%%========================================

%%========================================
%% Hyphenation
\hyphenation{pha-ses meta-program-ming resour-ces make-overs per-iods ar-ound data-base}
%%========================================

%%========================================
%% Start of document
%%========================================
\begin{document}
  %%----------------------------------------
  %% Information about the work
  %%----------------------------------------
  \title{\ProjectName}
  \author{\ProjectAuthor}
  \degree{Master in Informatics and Computing Engineering}
  %% Date of submission
  %\thesisdate{28$^{th}$ June, 2010}
  \thesisdate{28$^{th}$ June, 2010}

  %% Insert copyright text if used
  % \copyrightnotice{Gonçalo Santarém da Silva, 2010}

  \supervisor{Supervisor}{\ProjectSupervisor}{(PhD.)}
  %% Uncomment next line if necessary
  %% \supervisor{Second Supervisor}{Filipe Abrantes}{(PhD.)}

  %% Uncomment committee stuff in the final version
  \committeetext{Approved in oral examination by the committee:}
  \committeemember{\textbf{Chair}}{João Manuel Paiva Cardoso}{(Associate Professor)}
  
  \committeemember{\textbf{External Examiner}}{Alberto Manuel Rodrigues da Silva}{(Associate Professor)}
  \committeemember{\textbf{Internal Examiner}}{Ademar Manuel Teixeira de Aguiar}{(Auxiliary Professor)}
  \signature
  \committeedate{31$^{st}$ July, 2010}

  %% Specify cover logo (in folder ``figures'')
  \logo{feup-logo.pdf}

  %%----------------------------------------
  %% Cover page(s)
  %%----------------------------------------
  \maketitle
  %% Uncomment next line in the final version
  \committeepage

  %% Preliminary materials
  \StartPrelim
  \begin{singlespace}
    % TODO: Review at the end

\chapter*{Abstract}
\pdfbookmark[0]{Abstract}{abstract}

Web's popularity and importance on everyday life increases day by day. Its users have high expectations and require better user experiences in their daily interactions. Ruby on Rails was created as a tool to help coping with this demand. This framework is criticized for issues with scalability. The lack of generic guidelines, profiling tools and global awareness of the importance of building highly performant applications all contribute to this problem.

Producing the aforementioned generic guidelines, greatly improving Rails-related profiling tools and increasing this subjects' awareness inside the community are the main goals of this work.

System components have been addressed in three main phases: benchmark, tweak and develop, all from a Rails-centered perspective. Benchmarking operating systems involved comparing various Linux distributions and FreeBSD. Gentoo was the best performing alternative. As of tweaking, kernel configurations were explored. A generic benchmarking script for UNIX systems was developed.

Regarding Ruby interpreters, YARV outperforms and improves the scalability of Rails applications over MRI. Its garbage collector's flexibility was enhanced for adaptive performance. Its storage and retrieval capabilities of profiling information were extended and a graphical profiling output integrated. 

As of web servers, a memory usage monitoring script was developed. All web servers yielded similar performance results, although Thin had a remarkably lower memory consumption. Nginx was the best performing reverse proxy. Finally, many configuration options were explored. 

Regarding databases, a Ruby database library for MySQL was improved.

Concerning Rails, common pitfalls were exposed and solutions presented, exploiting their performance differences. Its profiling tools were improved and now seamlessly integrate with Ruby's. Redmine and many plugins were ported to the latest versions of Ruby and Rails. Finally, an article series on performance optimization was started, as well as the development of an official benchmarking suite for performance-oriented continuous integration.

The aforementioned generic guidelines and conventions were created. The native profiling tools of Ruby and Rails were refactored. Revamping the profiling tools, porting Redmine and plugins, writing an article series, adding flexibility to YARV's garbage collector and developing a benchmarking suite for continuous integration are activities which generally contribute to the awareness on this subjects' importance.

\chapter*{Resumo}
\pdfbookmark[0]{Resumo}{resumo}

A importância da Web aumenta diariamente. Os utilizadores têm expectativas elevadas, exigindo experiências de alta qualidade. O Ruby on Rails ajuda a colmatar estas exigências, embora se critique a sua escalabilidade. A falta de guias genéricos, ferramentas de análise de desempenho e consciência da importância do desempenho contribuem para este problema.

Os objectivos principais deste trabalho foram os de criar os referidos guias genéricos, melhorar as ferramentas de análise de desempenho existentes e aumentar a importância deste assunto.

Trabalharam-se os componentes em três fases: testes, configurações e desenvolvimento, sempre da perspectiva do Rails. Quanto a sistemas operativos, comparam-se várias distribuições de Linux e o FreeBSD. O Gentoo foi a alternativa com melhor desempenho. Várias opções do \textit{kernel} foram exploradas. Foi desenvolvida uma rotina de teste genérico de desempenho para sistemas UNIX.

Quanto ao Ruby, o YARV tem melhor desempenho que o MRI. Várias opções de parametrização foram introduzidas no seu \textit{garbage collector}. A sua capacidade de análise de desempenho foi aumentada e estendida.

No que toca aos servidores web, foi desenvolvida uma rotina de monitorização de memória. Todos eles evidenciam um desempenho similar, embora o Thin se destaque pela baixa utilização de memória. O Nginx obteve os melhores resultados como \textit{reverse proxy}.

Relativamente às bases de dados, uma biblioteca Ruby de MySQL foi melhorada.

Quanto ao Rails, vários problemas recorrentes foram analisados e foram propostas soluções. As ferramentas nativas de análise de desempenho foram melhoradas e integram, agora, com as do Ruby. A Redmine e vários \textit{plugins} foram portados para as últimas versões do Ruby e do Rails. Por fim, deu-se inicio a uma série de artigos sobre optimização de desempenho bem como se iniciou o desenvolvimento de uma bancada oficial de testes de desempenho para a integração contínua do Rails.

Criaram-se as linhas de guia genéricas referidas. Aprimoraram-se as ferramentas de análise de desempenho do Ruby e do Rails. Estas melhorias, a actualização da Redmine e \textit{plugins}, a série de publicações, a adição de flexibilidade ao YARV e o melhoramento da integração contínua do Rails são actividades que ajudam a aumentar a importância geral deste assunto na comunidade.

    \chapter*{Acknowledgements}
\section*{}
To everyone at \textit{Escolinhas.pt}, for allowing me to mix work and fun on a daily basis.\\
\mbox{}\\
To Ademar Aguiar and Nuno Baldaia, for their patience at guiding a sometimes slugabed student.\\
\mbox{}\\
To Muhammed Ali, for helping me set my priorities when coding tons of C was what I really wanted to do.\\
\mbox{}\\
To Brian Lopez, for happily sharing his knowledge on Ruby C extensions when all the documentation I could find was written in Japanese.\\
\mbox{}\\
To Rupak Ganguly, for helping me polish and publish my first magazine article ever.\\
\mbox{}\\
To Yehuda Katz, for guiding me through the Ruby Summer of Code and trusting me every single time, even when it meant breaking Rails.\\
\mbox{}\\
To Pedro Coelho, for lending me his computer when all official means have failed.\\
\mbox{}\\
To Paulo Pereira, for revising this report, pointing me in the right direction when I started and for continuously motivating me.\\
\mbox{}\\
To everyone who stained their black cloaks with me, for teaching me things I could not have learned elsewhere.\\
\mbox{}\\
To Sara, for all her love, help and support, for cheering me up whenever I felt down and for making me feel like I am the best person in the world.\\
\mbox{}\\
To my family, for their patience and support while I pulled dozens of all nighters during this awesome course, and for giving me the possibility to make the most out of it.\\

    \cleardoublepage
\thispagestyle{plain}

\vspace*{8cm}

\begin{flushright}
  \emph{``Fast isn`t a feature. Fast is a requirement.''} \\
  \vspace*{1.5cm}
  --- Jesse Robbins
\end{flushright}

%% Success is not about scale, it’s about sustainable execution
%% Jeff Putz
    % initial quotation if desired
    \cleardoublepage
    \pdfbookmark[0]{Table of Contents}{contents}
    \tableofcontents
    \cleardoublepage
    \pdfbookmark[0]{List of Figures}{figures}
    \listoffigures
    \cleardoublepage
    \pdfbookmark[0]{List of Tables}{tables}
    \listoftables
    \cleardoublepage
    \pdfbookmark[0]{Definitions}{def_defs}
    \chapter*{Definitions} % (fold)
\label{cha:definitions}
\headermark{Definitions}
The following definitions are in order of appearance.\\\\

\begin{description}
  \item[World Wide Web] System of interlinked hypertext documents contained on the Internet.
  \item[Coupling] The degree to which each program module relies on each one of the other modules.
  \item[Continuations] C-like gotos for Ruby.
  \item[Fork] A processes' act of creating a copy of itself.
  \item[Endian] Order of individually addressable sub-units within a longer data word in external memory.
  \item[Green threads] Threads scheduled by the Virtual Machine, emulating multi-threaded environments.
  \item[Mark and Sweep] The first garbage collection algorithm, consisting of two blocking phases: a mark phase and a sweep phase.
  \item[Just-in-time compilation] Also known as dynamic translation, just-in-time compilation is the act of converting code at runtime prior to executing it natively.
  \item[Ahead-of-time compilation] Compilation of intermediate code into a system-dependent binary.
  \item[EventMachine] A library for Ruby, C++ and Java that provides event-driven I/O using the Reactor pattern.
  \item[RDoc] The embedded documentation generator for the Ruby programming language.
\end{description}

% chapter definitions (end)

    \cleardoublepage
    \pdfbookmark[0]{Abbreviations}{def_abbr}
    \chapter*{Abbreviations} % (fold)
\label{cha:abbreviations}
\headermark{Abreviations}
The following abbreviations are in alphabetical order.\\\\

\begin{description}
  \item[AJAX] Asynchronous JavaScript and XML
  \item[API] Application Programming Interface
  \item[CERN] \textit{Conseil Européen pour la Recherche Nucléaire}, currently known as European Organization for Nuclear Research
  \item[CFQ] Completely Fair Queuing
  \item[CPU] Central Processing Unit
  \item[CSV] Comma-Separated Values
  \item[DBMS] Database Management System
  \item[DRY] Don't Repeat Yourself
  \item[ERB] Embedded Ruby
  \item[HTTP] Hypertext Transport Protocol
  \item[IP] Internet Protocol
  \item[I/O] Input/Output
  \item[JVM] Java VM
  \item[MB] Megabyte
  \item[MRI] Matz's Ruby Interpreter
  \item[MVC] Model-View-Controller
  \item[MPM] Multi-Processing Module
  \item[ORM] Object-Relational Mapping
  \item[OS] Operating System
  \item[PID] Process ID
  \item[PHP] PHP Hypertext Preprocessor
  \item[POSIX] \textbf{P}ortable \textbf{O}perating \textbf{S}ystem \textbf{I}nterface [for Uni\textbf{x}]
  \item[Rails] Ruby on Rails
  \item[RDBMS] Relational DBMS
  \item[REE] Ruby Enterprise Edition
  \item[STL] Standard Template Library
  \item[TCP] Transmission Control Protocol
  \item[UNIX] Uniplexed Information and Computing Service
  \item[USA] United States of America
  \item[VCS] Version Control Systems
  \item[VM] Virtual Machine
  \item[Web] World Wide Web
  \item[YARV] Yet Another Ruby VM
  \item[.NET] Microsoft .NET Framework
\end{description}

% chapter abreviations (end)



  \end{singlespace}

  %%----------------------------------------
  %% Body
  %%----------------------------------------
  \StartBody

  %% Introduction
  \chapter{Introduction} % (fold)
\label{cha:introduction}
\headermark{Introduction}

\section*{} % (fold)
This chapter briefly presents the project's context, purpose and scope. It covers the motivation behind it and its objectives.% while also detailing the report structure, providing an overview of each of the remaining chapters.

% section  (end)

\section{Context} % (fold)
\label{sec:context}
The internet started in the early 1990s as a work tool for CERN. It evolved into a vast information repository for the public~\cite{teaching_webdev_web20}---the Web---and had 16 million users back in 1995. Next year Netscape went public and achieved an impressive 90\% market share. It quickly lost its prominence during the first browser war, conceding its leadership to Microsoft's Internet Explorer~\cite{browser_wars}. The Web was starting to become a presence in everyday life. In 2009, 14 years later, the Web had more than 1700 million users and its popularity keeps growing nowadays~\cite{internet_stats}.

The Web is becoming an essential pillar of many businesses, social networks, gaming industries \textit{et al.} since distance is no longer an issue when it comes to information exchange.  It begins to present itself as a critical presence on computers nowadays. Many people believe future applications will mostly be web-based, pushing internet's importance even further~\cite{browser_application_platform}.

The growth of internet usage and its impact in corporate applications implies all kinds of particularities. First of all, users must trust the web. This is an essential pillar of any web service. To build costumer trust, service providers must pay attention to their users' needs and desires and they must meet their expectations~\cite{trust_semantic_web}. 

In the fall of 2001, many people concluded that the web was overhyped and bloated. There was the need for richer content and better user experiences~\cite{oreilly_web20}. In this context, the term \textit{Web 2.0} was born. Its meaning is not well defined but a commonly accepted definition states that it consists in the improvement of the first version of the Web~\cite{rubyonrails_tutorial} and involves many core concepts like usability and dynamic content~\cite{what_is_web20}.

Side by side with the Web's growth was the increasing user expectations when it comes to their experiences. As time goes by, users demand better interactions with the services they use. Their user experiences are partially based on response times~\cite{prioritizing_web_usability}, responsiveness and performance of Web applications. These concepts become part of the key factors to their success~\cite{responsiveness}. Users expect waiting times to be kept at an acceptable minimum and whenever they feel that this expectation is not being met their trust on the service diminishes. When users enter a given website they have a limited patience, related to their expectations and previous experiences. Whenever the system fails to meet those expectations, their patience decreases and it can cause them to leave. Steve Krug\footnote{Steve Krug is the author of the renowned book about human computer interaction and web usability entitled \textit{Don't Make Me Think}} entitles this phenomenon as the \textit{Reservoir of Goodwill}~\cite{dont_make_me_think}.

This increasing popularity, dependence and demand on the Web strives the developers for better tools to build quality web applications. Most developers seek the ability to increase their productivity while being able to build more complex, full-featured systems that suit their user's need, either it is a social or business-oriented service~\cite{comparison_agile_frameworks}.

With the growing importance of Web applications, many tools emerged trying to make the developer's life easier. \textit{Web 2.0} improved the internet and created a new set of needs and expectations. This need motivated the development of powerful frameworks and, among many others, the Ruby on Rails framework\footnote{\url{http://rubyonrails.org}} was born~\cite{what_is_web20}. As the community puts it~\cite{rubyonrails}:
\begin{quote}
  ``Ruby on Rails$^{\scriptsize \textregistered}$~is an open-source web framework that's optimized for programmer happiness and sustainable productivity. It lets you write beautiful code by favoring convention over configuration.''
\end{quote}
Ruby on Rails is one of the best-known Ruby frameworks~\cite{agile_webdevelopment_with_rails}. Most people, from individuals to companies, want a \textit{Web 2.0} killer application and Rails seems an excellent way of achieving it~\cite{oo_business_models}. This framework is commonly associated with the \textit{Web 2.0} concept, along with AJAX~\cite{spaghetti_code}. It is also deeply related with \textit{Agile Web Development}~\cite{agile_webdevelopment_with_rails}. Rails allows developers to build high quality applications with smaller effort---less time, less lines of code and less files, always with low coupling~\cite{maintainability_web_applications_j2ee_dotnet_ror}.

As Tim O'Reilly\footnote{Tim O'Reilly is the founder of O'Reilly Media and a supporter of the free software and open source movements} states~\cite{oreilly_ror}:
\begin{quote}
  ``Ruby on Rails is a breakthrough in lowering the barriers of entry to programming. Powerful web applications that formerly might have taken weeks or months to develop can be produced in a matter of days.''
\end{quote}
The huge success this framework had made it jump into a spotlight, with many companies starting to use it as the development framework for their applications. Some widely known services like \textit{Basecamp}\footnote{\url{http://basecamphq.com/}}, \textit{Twitter}\footnote{\url{http://twitter.com/}}, \textit{Hulu}\footnote{\url{http://hulu.com/}}, \textit{YellowPages}\footnote{\url{http://yellowpages.com/}} or \textit{GitHub}\footnote{\url{http://github.com/}} push this platform even further by giving it even more popularity~\cite{rubyonrails_applications}.
\begin{description}
  \item[Basecamp,] the original Rails application~\cite{rubyonrails_applications}, is an online project management tool which features live collaboration and task aiding software. It has over 1 million users since 2006~\cite{basecamp_turns_1000000}. It is ranked 516$^{th}$ on the Alexa Traffic Rank~\cite{alexa}.
  \item[Twitter,] a real-time short messaging service that works on multiple networks and devices. It is used for quick sharing of information, either updates from friends or breaking world news updates. It had more than 18 million adult users in the USA by the end of 2009 and is expected to achieve an impressive quantity of 26 million adult users in the same country by 2010~\cite{emarketer_twitter_usage}. It is ranked 11$^{th}$ on the Alexa Traffic Rank~\cite{alexa}.
  \item[Hulu,] a TV and Movie streaming website which allows users to watch their favorite videos on the browser for free. It had a significant amount of traffic in 2009, only staying behind Google services and Fox Interactive Media~\cite{hulu_growth}. It ranks 177$^{th}$ on the Alexa Traffic Rank~\cite{alexa}.
  \item[YellowPages,] a service that indexes and provides business listings of the United States of America, allowing users to search for services they are looking for, among other features. It ranks 929$^{th}$ on the Alexa Traffic Rank. To note that it is a USA-oriented application, raking 161$^{th}$ if the scope is limited to the USA~\cite{alexa}.
  \item[GitHub,] one of the best know repository hosting system which works with the Git VCS. It currently has more than 185 thousands of registered developers and it is associated with public, open source projects but also supports proprietary development. It ranks 997$^{th}$ on the Alexa Traffic Rank~\cite{alexa}.
\end{description}
These systems have high scalability demands. In order to keep increasing their popularity and to keep building users' trust, they also need to provide a great user experience along with reasonable response times while serving thousands of requests concurrently.

Some press reports question Rails' ability to scale, mainly based on the issues \textit{Twitter} faced when its growth reached a given magnitude~\cite{interview_alex_payne}. However, most of the issues were demystified as a software architecture design issue, taking the blame off of Ruby on Rails~\cite{ror_ecosystem_whitepaper}. Nonetheless, despite all the advantages this framework possesses, scalability is not one of them. Ruby is not as scalable as PHP or Java but on the other hand offers higher development speeds~\cite{issues_web_frameworks}. Luckily, only high traffic websites have to get deeply involved in scalability details but it is still an issue to be addressed.

Developers should be able to build Rails-based high-quality applications whose scaling is not directly related to hardware upgrades~\cite{interview_alex_payne}. Issues should be identified and solutions proposed so that this acclaimed Ruby framework becomes more scalable \textit{out of the box}, diminishing its dependence on hardware upgrades or major architectural changes. Developers should be aware of their choices' benefits and shortcomings. This way, Ruby on Rails development happiness can not only last through the creation of a web platform, but also through its maintenance.
% section context (end)

\section{Motivation and Objectives} % (fold)
\label{sec:motivation_and_objectives}
The Web starts to play a critically important role in many people's lives, either from a professional or personal point of view. User experience has become of great importance in recent times, with the \textit{Web 2.0} raising the expectations on a better interaction.

Internet accesses keep increasing in number and the recent developments in the smartphone, tablet and netbook's areas help increasing this number even further, with people starting to be permanently connected to the internet~\cite{npd:3g,mobileweb,netbooks}.

Users expect the Web to work as they preconceived and this fact has a great focus on recent developments in Interaction Design~\cite{interaction_design}. As innovation pushes user experiences to a new level, with richer information presented and organized in ways never seen before, technologies tend to emerge to support such evolutionary content and forms of organization. Developers need to meet the users' needs and they do not have unlimited time to do it, thus many recent frameworks have gained notorious popularity for being agile and robust, bringing productivity rates to a higher level~\cite{trends_webdev}. Ruby on Rails, as most recent frameworks, offers convenient methods and features which greatly improve the product's quality without the need for extended development times. However, it also makes it harder for the development team to build a highly scalable application when there are limited hardware resources~\cite{look_common_performance_problems_rails}.

Scaling and performance optimization should not be so hard to achieve in Rails, though. Many \textit{Web 2.0} platforms are created everyday and Rails-related scalability issues should not be an obstacle to their success. The framework's purpose is to help developing high-quality applications, not limiting their accomplishments to a given number of concurrent users. The Web should be able to shine in all of its glory and tools like Ruby on Rails are here to allow it to improve and innovate further and further, meeting the users' increasing standards and demands.

Performance optimization has been a work focus since computers were born~\cite{mass_memory_system_optimization}. Many people have focused on optimizing many different components, like the Ruby interpreter~\cite{yarv}, Rails itself~\cite{rails_merb_merge_performance} or the superjacent application~\cite{scaling_rails_bottomup,vaporware_to_awesome,rebuilding_scaling_yellowpages,5tips_scale_ror}. Many have focused on improving the speed and scalability of databases~\cite{performance_analysis_db_arch} and webservers~\cite{webserver_scheduling}. Others look for potential issues in the Operating System~\cite{unix_os_comparison, architecture_impact_os} and end up patching and tweaking the system's configuration. One can infer that most of the performance optimization activities focus on single elements or try to find out a single culprit to blame. It is also necessary to envision performance optimization as a globalist activity. If a small part of the system changes it will affect all those who interact with it by smaller or larger margins. In Chaos Theory this is called \textit{Sensitive Dependence} or, as more commonly known, the \textit{Butterfly Effect}. As mentioned in \textit{Quantum Chaotic Environments}~\cite{butterfly_effect_quote}:
\begin{quote}
  ``The exponential divergence (\ldots) from slightly different initial conditions---the famous butterfly effect---is a fingerprint of chaos in classical mechanics.''
\end{quote}
Rails is a highly dependent system. There are many components involved and all of them can be optimized. The key concept is that in order to improve a Rails application's scalability, the task should be addressed with a deep notion of its associated components. The whole system performance is what truly matters, not the performance of its individual parts. The core mindset of this project is to address all involved components from Rails' perspective.

There is need for a solid set of generic development conventions and guidelines oriented towards scaling and performance. Developers seek optimal configurations for all the components involved so they can wring every processing cycle out of their applications, in order to increase scalability and decrease response processing times. There is urgency in looking at all the tools Rails depends on and optimizing them to suit Rails' needs---the system, envisioned as a whole.

This philosophy will be moslty applied in \textit{Escolinhas}\footnote{\url{http://escolinhas.pt}}, a rapidly growing Portuguese Ruby on Rails based project. \textit{Escolinhas} aims at sustaining social and collaborative work for children in elementary schools involving students, teachers and parents as its users. With the user demand increasing day by day, it becomes an excellent case-study application to research, test and apply all the work and discoveries made during the course of this thesis.

The goals of ``Scaling Rails: a system-wide approach to performance optimization'' are to produce the aforementioned generic conventions and guidelines, to find generalist optimal configurations for the most important components that Rails depends on and to optimize them for the framework at use, all in the \textit{Escolinhas} context mentioned before. Other goals include improving the current profiling tools and increasing this subject's relevance amongst the Ruby on Rails community.
% section motivation_and_objectives (end)


% TODO: Review at the end
\section{Report Overview} % (fold)
\label{sec:report_overview}
The rest of this report is structured as follows.
\begin{description}
  \item[Chapter~\ref{cha:technologies}: ``Technologies''] gives an overview over this research's possibly involved technologies, providing important background information on each one of them.
  \item[Chapter~\ref{cha:state_of_the_art}: ``State of the Art''] reviews each component's performance and scalability when compared to its alternatives along with a few other details.
  \item[Chapter~\ref{cha:approach}: ``Approach''] suggests a possible approach into tackling the presented issue.
  \item[Chapter~\ref{cha:conclusions}: ``Conclusions''] narrows alternatives down by using related work's conclusions, exposed in the state of the art, to discard certain options.
\end{description}
% section report_overview (end)

% chapter introduction (end)

  %% Technologies review
  \chapter{Technologies Review} % (fold)
\label{cha:technologies_review}
\headermark{Technologies Review}

\section*{} % (fold)
Since this research spans into multiple traditional fields, this chapter provides an overview of the technologies involved. A high-level analysis is made concerning their most important characteristics and a solid knowledge base is provided as a sustaining base for exposing their state of the art, further analysis and development.
% section  (end)

\section{Operating Systems}
\label{tech:sec:operating_systems}
The operating system is the base of everything else. It runs on top of the hardware and allows applications to use its resources. There are many kinds of operating systems, with different characteristics and philosophies. The relevant ones to this research will be explained in the following sections.

\subsection{GNU/Linux}
Linux is the term used to describe UNIX-based operating systems that run the Linux \textit{Kernel}. It was created in 1991 by Linus Torvalds~\cite{linux_kernel_evolution}, who is also the author of \textit{git}~\cite{pro_git}. It is one of the most significant open-source projects where volunteers from all over the world work together to achieve a common goal---improve the operating system itself~\cite{linux_kernel_evolution}. While having low usage on desktop systems~\cite{w3counter}, Linux is widely used on servers, mainframes and super computers. It is commonly known for its security and reliability. One sustaining example is the list of operating systems used by the most reliable hosting companies in December 2009, where Linux figures 6 times in the top 10~\cite{netcraft_dec2009}.

Linux stands as the base for many UNIX-like software distributions, specifically Linux distributions. These consist on a large collection of software applications and configurations which range from full-featured desktop systems to minimal environments. Non-commercial Linux distributions commonly used in server environments include:
\begin {description}
\item[Debian,] maintained by a developer community with a strong commitment to free software principles\footnote{\url{http://www.debian.org/}}.
\item[Ubuntu Server,] derived from Debian and maintained by Canonical, being the server version of the most popular Linux distribution\footnote{\url{http://www.ubuntu.com/products/whatIsubuntu/serveredition}}.
\item[CentOS,] derived from the same sources used by the renowned \textit{Red Hat}\footnote{\url{http://www.centos.org/}} distribution.
\item[Gentoo,] known for its FreeBSD Ports-like system for custom compiling\footnote{\url{http://www.gentoo.org/}} of applications.
\end{description}
Each distribution aims at adding its own flavor to the Linux operating system, providing different user experiences to their users~\cite{tuning_costumizing_linux}.


\subsection{Berkeley Software Distribution}
Berkeley Software Distribution, also know as BSD or \textit{Berkeley} \textit{Unix}, is considered to be a branch of the \textit{Unix} operating system. It was created in 1977 in the University of California in Berkley by the Computer Systems Research Group.  Entitled by some as the greatest software ever written~\cite{ greatest_software_ever_written}, even Linux's creator, Linus Torvalds himself, went as far as stating~\cite{ interview_linus}: 
\begin{quote}
  `` If 386BSD had been available when I started on Linux, Linux would probably never had happened''
\end{quote}
Just like Linux, it is open-source software and has several associated distributions, although at a smaller scale. FreeBSD\footnote{\url{http://www.freebsd.org/}} is the most popular BSD distribution~\cite{top_5_bsd_distros}.

A significant remark is that both Apple's \textit{Mac OS X} and Microsoft's \textit{Windows} use parts of FreeBSD's source code~\cite{leopard_os_foundations,bsd_code_windows}.

\subsection{Microsoft Windows Server}
Windows Server is Microsoft Corporation's operating system oriented towards servers. The current version is entitled \textit{Windows Server 2008} and, as its name implies, was released in 2008. It is built on top of the same code base used in \textit{Windows Vista}.

Windows is proprietary software and consequently does not come in a distribution manner like the \textit{UNIX}-based operating systems mentioned before.

\subsection{Mac OS X Server}
Mac OS X Server is Apple's server-oriented operating system. It is architecturally identical to its desktop counterpart, except that it includes work group management and administration software tools.

This operating system is usually found on rack mounted server computers which are also designed by Apple.

\section{Ruby} % (fold)
\label{tech:sec:ruby}
Ruby is a dynamic and object-oriented language created by Yukihiro Matsumoto, who released it to the public in 1995. The purpose was to create a ``language that was more powerful than Perl, and more object-oriented than Python''~\cite{interview_creator_ruby}.

Ruby was inspired by languages such as Lisp, Smalltalk and Perl, and its core characteristics and features include~\cite{ruby_about, ruby_book}:
\begin{description}
  \item[Open source.] Ruby's license allows anyone to use, copy, modify or distribute it.
  \item[Pure object-oriented.] In Ruby, everything is an object, including classes, modules and data types---even numbers, booleans and null values which in other object-oriented languages are known as ``primitives''. An object's properties are known as ``instance variables'' and actions are ``methods''.
  \item[Flexible.] It not only features dynamic typing, but also very powerful reflective and metaprogramming capabilities. Ruby does not restrict what a programmer can do. Any part of Ruby code can be removed, redefined or extended, even at runtime. This is not only true for a programmer's own code, but also for core Ruby classes such as Object and String.
  \item[Automatic memory management.] Like other languages such as Java and contrary to C, developers do not manage the program's memory usage.
  \item[Portable.] Ruby is mainly developed in GNU/Linux but can run on most operating systems and platforms such as BSD, Mac OS X, Windows 95 and many others.
  \item[Exception handling.] Ruby can recover from errors just like Java, C++ and Python.
\end{description}
Ruby started to become popular in 2001, with the start of the~\textit{RubyGems}\footnote{\url{http://rubygems.org/}} project, which allows easy packaging and distribution of applications and libraries~\cite{railssolutions}. Ruby has two main versions, 1.8 and 1.9, the last one having been publicly released approximately two years ago~\cite{ruby191_release}.


\subsection{Ruby 1.8}
The most commonly found version of Ruby in Rails' projects is the 1.8, mainly due to the fact that it was the officially recommended version for many years.

\subsubsection{MRI}
Ruby's 1.8 official interpreter was developed by the language's creator---Yukihiro Matsumoto, also known as \emph{Matz}---and its first public release happened in 1995. Some people mistake MRI for ``Main Ruby Implementation'' but the abbreviation is actually related to its creator's name---\emph{Matz Ruby Interpreter}. Some particular characteristics of this interpreter include:
\begin{description}
\item[Language.] MRI is written in C.
\item[Threading.] It can emulate a threaded environment without relying on the operating system capabilities by using \textit{green threads}.
\item[Garbage Collector.] The garbage collector is based on the simple \textit{mark-and-sweep} algorithm.
\item[Extensions.]  Developers can extend Ruby's basic functionality by writing their own extensions. These can be written in Ruby or in native C, by using MRI's powerful API.
\item[Bytecode Interpretation.] MRI lacks a bytecode interpreter. When a program is executed, it parses its source and creates its syntax tree. Then, it iterates over this tree directly while executing the program.
\end{description}
MRI was the only official Ruby interpreter for many years and, consequently, it is the most widely used.


\subsubsection{Ruby Enterprise Edition}
This Ruby interpreter, called REE, is based on MRI's source code. However, it includes many Rails-oriented enhancements. It was first released in 2003 and has been merging with MRI periodically, keeping its own changes aside. Its characteristics and differences from the official interpreter for version 1.8---MRI---include~\cite{rubyenterpriseedition}:
\begin{description}
\item[Language.] Since REE is based on MRI, it is written in C.
\item[Threading.] The thread implementation is the same found on MRI.
\item[Garbage Collector.]  The garbage collector was improved, being \textit{copy-on-write} friendly. This allows for reduced memory usage when paired with Passenger, who uses \textit{preforks} in combination with this feature.  It also enables the user to tweak its settings for adaptive performance.
\item[Extensions.]  Just like MRI, it natively supports Ruby and C extensions. 
\item[Bytecode Interpretation.] REE is mostly based in MRI's source code, as mentioned before, so it lacks a bytecode interpreter. It also iterates over the program's syntax tree directly.
\item[Other Differences.] REE uses \textit{tcmalloc} for memory allocation, which improves the process's performance. It also allows better debugging by introducing the ability to inspect the garbage collector's state and to dump stack traces for all running threads.
 \end{description}
This Ruby interpreter is normally paired with \textit{Phusion}'s\footnote{\url{http://www.phusion.nl/about.html}} web server---Passenger. Since they are developed by the same team, some optimizations are more noticeable when these two are being used together.


\subsubsection{JRuby}
JRuby is a Java implementation of Ruby whose first version to support Rails was released in 2006. It has many differences from the MRI and these include~\cite{mri_jruby_comparison, jruby_performance_glassfish}:
\begin{description}
\item[Language.] JRuby is written in Java, running on top of JVM.
\item[Threading.] The thread implementation is based on JVM threads. These are more efficient than the green threads used by MRI, since they are native threads.
\item[Garbage Collector.]  The garbage collector is inherited from Java, being generational-based. It also inherits its heap and memory management, granting it greater performance since Java's memory management is overall very efficient.
\item[Extensions.]  JRuby does not support native C extensions, but the most popular ones have already been ported to Java.
\item[Bytecode Interpretation.] In this concern, JRuby has the advantage of running on top of the JVM. The Ruby code can be interpreted directly, like MRI's behavior, but it can also be targeted for \textit{just-in-time} or \textit{ahead-of-time} compilation to Java bytecode, which is handled very efficiently by the JVM.
\item[Other Differences.] Continuations and forks are not supported in JRuby. There are also a few small differences around its native \textit{endians} and time precision. It does, however, support the Ruby 1.8 specification to full extent and is currently used in many production environments.
\end{description}


\subsection{Ruby 1.9}
This version represents a step forward in the Ruby programming language. It includes many new features and enhancements. Some of these include \textit{Fibers} and \textit{Non-blocking I/O} improvements~\cite{changes_ruby19}.

\subsubsection{YARV}
Yet Another RubyVM, also known as YARV, has been adopted as MRI's successor, becoming the official interpreter for version 1.9. For this same reason some people call it KRI or \textit{Koichi's Ruby Interpreter}. It consists on a bytecode interpreter designed specifically for Ruby and it is the only to fully support version's 1.9 specification. Its author purpose, upon its creation, was to reduce execution times of Ruby programs and it differs a bit from other implementations~\cite{yarv, rubyvm_interview, ruby_intermediate_language}:
\begin{description}
\item[Language.] YARV was developed in C and reuses many parts of its predecessor, like the Ruby script parser, the object management mechanisms and the garbage collector.
\item[Threading.] Contrary to MRI, YARV supports native threads. It also efficiently supports \textit{Fibers}, previously called \textit{Continuations}, without suffering from serious performance issues like its predecessor~\cite{memory_leak_fix_18X}.
\item[Garbage Collector.]  As aforementioned, YARV reuses MRI's code for its garbage collector.
\item[Extensions.]  YARV supports its predecessor's extensions either they are written in Ruby or C. This represents, however, a bottleneck on parallel computing because of existing extensions' synchronization issues.
\item[Bytecode Interpretation.] As mentioned before, the main difference between MRI and YARV when it comes to performance is related to the last's generation of intermediate code, which is much faster to process than parsing the program's syntax tree nodes one by one.
\item[Other Differences.] There are many differences between YARV and MRI, the reference interpreter. An important one is the usage of \textit{Fibers} which allows the developer to do cooperative scheduling instead of using the preemptive context switch model, commonly used in thread scheduling. On the same subject, \textit{Fibers} are cheaper to create than threads. 
\end{description}
YARV initially had a great impact in the Ruby community for its enhancements. It brings many advantages as new features, better performance and improved memory usage. However, it has not been extensively adopted because of some existing incompatibilities with some libraries which have not been upgraded to comply with Ruby's new specification~\cite{rubys_challenge_2009}.


\subsection{Promising interpreters in heavy development}

There are many \textit{work-in-progress} implementations of interpreters of the Ruby programming language. Some of them are explored in this section.

\subsubsection{Rubinius}

Rubinius has a great deal of focus on performance and its features include support for native POSIX threads, a generational garbage collector, compatibility with MRI/YARV extensions and a more efficient bytecode compiler~\cite{rubinius, rubinius_virtual_machine, rubinius_virtual_machine_rewrite}. Unfortunately Rubinius only supports 93\% of the Ruby specification, according to the pass rate test from \textit{RubySpec}\footnote{\url{http://rubyspec.org/}}. However, this interpreter is very promising and its compatibility with the language specification is constantly growing.


\subsubsection{MacRuby}

MacRuby is mainly based on YARV. By using Objective-C's engine, it includes a generational garbage collector and supports native POSIX threads. While being a promising Ruby interpreter specifically designed for the Mac OS X operating system, it has yet to achieve an acceptable degree of compatibility with Ruby's specification, which is currently at 85\%. Unfortunately, the current version is not able to run a standard Rails 3 application without specific modifications~\cite{macruby_06} but a great deal of effort is being put into increasing its compliance with Ruby's specification.

\section{Rails Web Servers} % (fold)
\label{tech:sec:rails_webservers}
In its simplest manner, a web server is a never ending \textit{loop} that accepts connections on a listening \textit{socket} and handles them somehow. There are notorious differences on how this \textit{loop} is implemented, besides the classical architectural and philosophical differences behind each web server. These differ in handling multi processing, multi threading, asynchronous events, data copying, context switching, locking contention, memory management, blocking operations, HTTP parsing, the TCP stack implementation and many other architectural differences.


\subsection{WEBrick}
This is Ruby's pioneer web server. It was created in 2000 by Masayoshi Takahashi and Yuuzou Gotou. WEBrick is a full-featured server that supports HTTP, HTTPS and listening concurrently to several ports, among other features. It is purely written in Ruby and has a very modular design, allowing developers to extend its functionalities by supporting external handlers~\cite{webrick_guide}.
WEBrick uses a single process but spawns a new thread for each incoming request. Mainly due to being written in Ruby, its HTTP parser is known for its poor performance~\cite{ruby_webservers}. WEBrick's request handling is demonstrated on figure~\ref{fig:webrick_architecture}.
\begin{figure}[h!t]
  \centering
    \includegraphics[width=0.75\textwidth]{webrick_architecture}
    \caption{WEBrick's Request Handling Process} \label{fig:webrick_architecture}
\end{figure}
Due to its poor performance, users normally used alternate, less conventional setups which were also known for their poor stability~\cite{ruby_webservers}.


\subsection{Mongrel}
Mongrel was released by Zed A. Shaw in 2006 and soon became the most popular web server used to run Rails applications. It offered a much better performance when compared to WEBrick and it was reasonably suited for production environments. This was mainly due to its improved implementation of the HTTP parser, which was rewritten in C~\cite{mongrel_server_production}.

Similarly to WEBrick, Mongrel uses a single process. It has an acceptor thread which handles incoming connections, launching a new thread for each one of them. In production environments, Mongrel is commonly found in clustered configurations where several processes are launched and their usage is dictated by a proxy server~\cite{mongrel_faq}. Mongrel's request handling is demonstrated on figure~\ref{fig:mongrel_architecture}.
\begin{figure}[h!t]
  \centering
    \includegraphics[width=0.75\textwidth]{mongrel_architecture}
    \caption{Mongrel's Request Handling Process} \label{fig:mongrel_architecture}
\end{figure}
Mongrel also optimized the TCP stack by changing Ruby's default socket listening queue from 5 to 1024, besides using optimization flags on socket connections to improve bandwidth usage~\cite{mongrel_faq}.


\subsection{Thin}
Thin was released in 2008 and was the first Ruby web server which did not follow the \textit{one thread per request} convention. It uses Mongrel's HTTP parser and EventMachine as its I/O back-end, allowing it to use a fast asynchronous event loop in a single thread for all incoming requests~\cite{thin}. Thin recently became able to combine threading with its philosophy, by allowing the creation of a background pool of 20 threads~\cite{ruby_webservers}. Thin is written in C, C++ and Ruby and is optimized for small requests and fast clients. Its request handling is demonstrated on figure~\ref{fig:thin_architecture}.
\begin{figure}[h!t]
  \centering
    \includegraphics[width=0.75\textwidth]{thin_architecture}
    \caption{Thin's Request Handling Process} \label{fig:thin_architecture}
\end{figure}
This setup yields better performance and scalability than Mongrel, especially when serving small requests like, for example, API calls. This is mainly related to the fact that this web server does not launch a new thread for each request, requiring less memory and no context switches~\cite{ruby_webservers}.
 

\subsection{Passenger}
Passenger was also released in 2008 but it had a big difference from the other alternatives since it was not a self-contained web server. Instead, it makes use of established web servers like Apache or Nginx by using their reliable web stack. It is mostly written in C++ and used as a module or extension to these general purpose web servers, adding the needed functionality to support Ruby and handling certain types of requests~\cite{passenger_whatis}.

When the main web server starts, having Passenger loaded as a module, it launches a Ruby process that will be responsible for all the other processes handling the Ruby application, called ``worker processes''. Each request is delivered to the firstly created Ruby processes---the master process---which forwards it to one of its workers. These worker processes are single threaded and handle one request at a time~\cite{ruby_webservers}. Passenger's request handling is demonstrated on figure~\ref{fig:passenger_architecture}.
\begin{figure}[h!t]
  \centering
    \includegraphics[width=0.75\textwidth]{passenger_architecture}
    \caption{Passenger's Request Handling Process} \label{fig:passenger_architecture}
\end{figure}
It is the first real multi-process server for Ruby, although setups with multiple Mongrel or Thin processes behind a reverse proxy were already being used~\cite{passenger_whatis}. Passenger is a free, open-source product but \textit{Phusion} also provides commercial support.


\subsubsection{Apache}
The Apache HTTP Server is a full-featured and open source web server created by the \textit{Apache Software Foundation}. It consists on a general purpose web server and provides many useful features such as HTTPS, IPV6 and authentication. Apache natively handles many languages such as PHP and Perl~\cite{apache_features}. It can be extended with modules and this is where Passenger comes in---it will act as \textit{mod\_rails} and extend Apache's functionality to be able to handle Ruby on Rails applications~\cite{passenger_whatis}.


\subsubsection{Nginx}
Nginx is a general purpose lightweight open source web server with a strong focus on performance~\cite{nginx_features}. It was created by Igor Sysoevy and Passenger can extend its functionality by being installed as a module, similarly to the Apache's procedure~\cite{passenger_whatis}.


\subsection{Unicorn}
Unicorn's first stable version was released in 2009. It is a self-contained web server designed to take advantage of Unix-based kernels and is optimized for fast clients with low latency~\cite{unicorn}. It delegates every task that is better supported by the operating system, Nginx or Rack to themselves, respectively. It uses one master process that spawns and reaps a user-defined number of worker processes without any thread usage. One of its main features is that load balancing is done entirely by the OS kernel, avoiding that requests pile up behind a busy worker. Unicorn is written in Ruby, except for its HTML parser which is based on Mongrel's and, consequently, is written in C. When used in a production environment it should be deployed in conjunction with a reverse proxy capable of fully buffering both the requests and responses between itself and a slow client.
Unicorn's request handling is demonstrated on figure~\ref{fig:unicorn_architecture}.
\begin{figure}[h!t]
  \centering
    \includegraphics[width=0.75\textwidth]{unicorn_architecture}
    \caption{Unicorn's Request Handling Process} \label{fig:unicorn_architecture}
\end{figure}

\section{Databases} % (fold)
\label{state:sec:databases}
The database is responsible for persisting data. It stores objects information so they can be retrieved latter without needing to use the system's memory.
The database usually represents a worrying bottleneck for high performance applications. It interacts with the disk --- the system's main memory --- and in recent hardware this represents a performance bottleneck~\cite{memory_wall}. Accessing the disk is a costly procedure and web applications need a very efficient database in order to diminish this bottleneck's impact.

Across most relational databases, from MySQL to PostgreSQL and including proprietary DBMS, MySQL consistently outperforms all others~\cite{benchmark_relational_databases}.
Despite being the fastest schema-based database in production, a schema-less database --- MongoDB --- improves over MySQL's performance significantly. This kind of database is more flexible but it also uses more disk space, as row names are stored along with their content to enable this flexibility. MongoDB is not atomic, however, not enabling useful features like transactions ~\cite{mysql_to_mongodb}. This is not related to the fact of being a schema-less database but with the databases' philosophy, since CouchDB --- another well-known schema-free database created by the Apache Foundation --- supports transactions on a performance penalty.

A critical procedure to improve a system's scalability is database caching. The reduction of disk accesses provides huge performance improvements~\cite{scaling_rails_bottomup} allowing improvements grater than 100\%~\cite{rapid_prototyping_mdd,high_performance_database_caching}.


\section{Ruby on Rails} % (fold)
\label{tech:sec:ruby_on_rails}
David Hansson began developing a web-based project management tool oriented towards small teams in 2003. Working at \textit{37signals}, he initially started by using PHP but soon became frustrated by many of the language's shortcomings. He gave up on this language and started implementing what today is known as \textit{Basecamp} in pure Ruby. While developing the application, he noticed that a lot of its code could be extracted into a framework for future use with other applications. Hansson decided to release his framework to the public in July 2004 and Ruby on Rails was born~\cite{railssolutions}.

Rails started to become more mature over time and applications like Twitter, YellowPages, Hulu, Scribd and GitHub were built using this framework. Its popularity has grown significantly since the beginning and its adoption by popular platforms helped establishing Rails as a solid framework.

Ruby on Rails has three main principles which motivated its creation~\cite{agile_webdevelopment_with_rails, ruby_on_rails_principles}:
\begin{description}
\item[Convention over configuration.] In Rails, everything has a default configuration. The only exception is the database connection data. This way, developers only need to specify when they want to use unconventional configurations. This way, Rails offers simplicity while retaining high flexibility.
\item[Don't Repeat Yourself.] Also known as DRY, this practice implies that the similar code snippets do not exist in separate locations. Every piece of knowledge is unique, definite and has a relevant representation. This simplifies modifications by the avoidance of having to change the same logic in different parts of the project, allowing the applications to keep a high consistency degree.
\item[Model-View-Controller.] Rails follows the MVC architecture pattern, keeping the source code well organized by clearly separating the code according to its purpose. The \textit{Model} is responsible for maintain the state of the application, specifying the constraints its related data has to obey to. The \textit{Controller} receives the users' input, interacts with the model and finally renders a view page as the result. The \textit{View} can have multiple formats, from JSON to XML, and is essentially what is displayed to the users. This principle's schema is presented in figure~\ref{fig:mvc}.
\begin{figure}[h]
  \centering
    \includegraphics[width=0.75\textwidth]{mvc}
  \caption{Model-View-Controller architectural pattern}
  \label{fig:mvc}
\end{figure}\\
\end{description}
Rails' functionality can be altered and extended with \textit{plugins} and \textit{gems}. While being a full stack web framework, Rails does not aim to include every single feature. However, it has been built with a highly extensible infrastructure and it has a considerably large set of \textit{plugins} and compatible \textit{gems} nowadays~\cite{rails_magazine_1}.


\subsection{Rails 2}
\label{tech:sec:ruby_on_rails:rails2}
Rails 2 was first released in 2007 and is currently in version 2.3, released at the beginning of 2009. Throughout these 2 years the framework was improved by many contributors, aside from the core team~\cite{rails_core_team}. The framework is essentially divided in six essential modules~\cite{ruby_on_rails_principles, rails23_release_notes}:
\begin{description}
\item[ActionPack] splits the response in two:  a request for the controller to handle and a template rendering part for the view to control.
\item[ActiveRecord] is responsible for object handling and their database representations. Objects are directly linked to the database, so modifying them will modify the table definition they are associated with.
\item[ActiveResource] corresponds to objects that represent the application's RESTful resources as manipulatable Ruby objects.
\item[ActiveSupport] is a collection of various utility classes and standard library extensions.
\item[ActionMailer] is a framework for designing email-service layers, allowing the application to send emails using a mailer model and views.
\end{description}
Having a considerable amount of modules and classes, Rails is structured according to the aforementioned components.


\subsection{Rails 3}
\label{tech:sec:ruby_on_rails:rails3}
Rails 3 is currently in development and it is on its first release candidate, after four beta versions. Most of Rails' code has been refactored and this release's main goals were concerned with improved component decoupling and performance~\cite{rails3_great_decoupling}. 

As of component decoupling, a great deal of work has been done and impressive goals have been achieved~\cite{vaporware_to_awesome}. Most of Rails' components are agnostic now, having standard interfaces for communication with each other. The key concept is that a component is agnostic to whom it is interacting with. This allows component swapping, enabling the replacement of one or more of Rails' core components with a different, third-party one. In order to make this happen standard procedures have been developed, providing standard interfaces for each one of Rails' components.

The decoupling process also allowed for improved modularity, permitting Rails' component separation. ActionController, for instance, has been split into ActionDispatch, ActionController and AbstractController~\cite{vaporware_to_awesome}. There was a lot of work on explicitly handling each component's internal dependencies. This enables the developer to carefully select which modules he needs in his Rails application without caring about including the modules it depends on as well. In previous versions of Rails developers would import the top-level modules, since the alternative was to parse the source code of the framework to find its internal dependencies in order to import all necessary modules. Applications now have the possibility to only load the modules they really need thus becoming faster and lighter.

This improved modularity also had its impact on performance. However, the team also made a specific effort into improving common Rails bottlenecks like partial and collection rendering~\cite{vaporware_to_awesome}, among some other optimized sections.


% chapter technologies (end)

  %% State Of The Art
  \chapter{State of the Art in Rails Components' Performance} % (fold)
\label{cha:state_of_the_art}
\headermark{State of the Art}

\section*{} % (fold)
Each component's state of the art will be presented in this chapter, providing an analysis of related work.
% section  (end)

\section{Operating Systems} % (fold)
\label{state:sec:operating_systems}

This research is focused on server environments since deployed Ruby on Rails applications are commonly found in these setups. When it comes to servers, thought, the emphasis on performance becomes much more critical since it is a very important requirement that applications demand from this kind of environment. Some facts suggest that the most commonly found performance bottlenecks in servers are related to the operating system. Operating systems performance bottlenecks usually reside in three key issues~\cite{os_performance_server}:
\begin{itemize}
\item Process management;
\item Virtual memory management;
\item High performance I/O.
\end{itemize}
Each operating system addresses bottlenecks differently and their decisions impact application-specific and system-wide performance.

In this research's context, performance is addressed from a Ruby on Rails perspective. Windows' support for this framework and related components is poor and this environment makes the framework slower. Passenger, one of the major Rails web servers used in production, opted not to support the Windows platform. As Marcus Koze explains~\cite{marcus_koze_passenger}:
\begin{quote}
  ``We have no plans to port Passenger on Windows. Windows lacks the proper facilities to implement Passenger efficiently. Passenger on Windows will be very, very inefficient, which can give both Ruby on Rails as well as Passenger a bad name.''
\end{quote}
Many other web-oriented benchmarks show that Linux has better scalability and performance compared to Windows when it comes to this type of servers~\cite{apache_tomcat_performance_linux_windows, php_apache_linux_windows}. The difference is quite noticeable: Ruby on Rails is still more efficient in a Virtual Machine running Linux than in a normal installation of Windows~\cite{linux_virtualbox_windows_rails}.

Documentation on how to use and deploy Rails applications in the Windows platform is also scarce, often needing unconventional methods for being able to run it properly~\cite{rails_windows}.

Since Rails is built on top of the Ruby programming language, this framework's lack of performance in Windows is possibly related to the poor Ruby performance in this operating system. The Ruby 1.8 official interpreter is twice as fast in Linux than its Windows counterpart. As of YARV, the Ruby 1.9 official interpreter seems to be 70\% faster on Linux when compared to its Windows counterpart~\cite{ruby_faster_linux}. Windows-specific Ruby interpreters do not seem to have noticeable performance improvements~\cite{ruby.net} and lack the necessary stability and compliance with the Ruby specification. \textit{IronRuby}, for instance, is an interpreter developed by Microsoft~\cite{ror_ecosystem_whitepaper} that is faster than Windows' MRI but still slower than the YARV implementation for this operating system. It also presents stability issues, as timeouts are common in a considerable number of benchmarks~\cite{ironruby_performance}.

On a side note, MySQL's documentation states that there are a few unpleasent details when using this database under a Windows environment~\cite{mysql_windows_linux}. First of all, it is only conceded 4000 ports which after being used need two to four minutes before becoming available again. It also has a limited allowed opened files number---2048---restraining its concurrent capabilities. Finally, MySQL uses a blocking read for each connection and Windows' connection error handling is quite poor when compared to Linux's counterpart. As the aforementioned documentation states, all these differences limit the number of acceptable concurrent requests handled and difficult high-load handling. They can lead to performance issues and limit the system's scalability.

To date, there are no specific tests analyzing Ruby on Rails' performance on BSD when compared to other operating systems. However, many web discussions suggest that it is quite similar to the one achieved on Linux which is understandable since they are both \textit{UNIX} derivates.

\section{Ruby} % (fold)
\label{state:sec:ruby}
The Ruby on Rails framework, as its name implies, is built on top of the Ruby programming language. Rails' performance is highly affected by Ruby and the interpreter choice can lead to great impact on the framework's scalability.

Ruby has many interpreters according to its version. Ruby 1.8 has MRI as its default interpreter and Ruby 1.9 has YARV. There are, however, a few other implementations that start to deserve some attention. These present different philosophies and implementation details that provide them with a few advantages and other shortcomings. The most popular Ruby interpreters for Ruby 1.8 are as follows:
\begin{description}
\item[MRI,] the standard Ruby 1.8 interpreter.
\item[Ruby Enterprise Edition,] based on the MRI's code but modified for better performance in Rails environments.
\item[JRuby,] a Ruby interpreter built on top of JVM.
\item[Rubinius,] a promising Ruby interpreter built by Rails' creators --- \textit{EngineYard}.
\end{description}
Version 1.9 came out recently so alternative interpreters do not fully support it just yet. YARV, the standard Ruby 1.9 interpreter, is the only one to support the specification.

The default implementation for Ruby 1.8, MRI, is generally known by its poor performance~\cite{6tips_for_mri}. This motivated the development of efficient interpreters for the language. Ruby Enterprise Edition has better scaling abilities than MRI, generally performs better and uses less memory~\cite{ree_benchmarks}. JRuby outperforms MRI by a considerable margin~\cite{ruby19_performance}. Rubinius, on the other hand, also effortlessly outperforms MRI. It also performs better than JRuby in most cases but by a much smaller margin~\cite{rvm_rubinius_benchmarks}.

Some people found significant performance improvements in MRI by applying a series of patches and configurations to its garbage collector. Twitter benefited from this patches, roughly improving its overall performance by 30\%~\cite{ruby_gc_tuning}.

Ruby 1.9 has only one compliant interpreter as aforementioned. However, as reference benchmarking, its performance is better than JRuby's~\cite{ruby19_performance}  and is paired with Rubinius'. It performs slightly better in some benchmarks but also performs faintly worse on others~\cite{rvm_rubinius_benchmarks,ruby_interpreter_benchmarks}. Some people even got as far as saying it was more efficient than Python's counterpart~\cite{ruby19_python}.


\input{state_rails_web_servers}
\section{Databases} % (fold)
\label{tech:sec:databases}
A database collects data, either records or files. From the Rails perspective, this can either be a schema-based or a schema-less database.


\subsection{MySQL}
MySQL is the most popular open source database in the world, having consistently fast performance, high reliability and ease of use~\cite{why_mysql}. It was first released in 1995 and it is the default database on a Ruby on Rails project. This is the database of choice for all \textit{37signals}'s applications~\cite{interview_dhh}.  It is a relational schema-based database that offers useful features like various storage engines, transactions, indexes, load balancing and so on.

Figure~\ref{fig:mysql_architecture} presents a simple overview over MySQL's architecture.
\begin{figure}[h]
  \centering
    \includegraphics[width=0.75\textwidth]{mysql_architecture}
  \caption{Logical view of MySQL's architecture}
  \label{fig:mysql_architecture}
\end{figure}
The top layer is related with the services that are not unique to MySQL, like connection handling, authentication and security. The middle layer refers to crucial MySQL features like query parsing, analysis, optimization, caching and all the built-in functions.  This layer also holds all functionality across storage engines and stored procedures. Finally, the bottom layer consists in the storage engines themselves, responsible for the storage and retrieval of all stored data~\cite{high_performance_mysql}. MySQL has four main storage engines, all with advantages and disadvantages, the default one being InnoDB.

A high-level summary of the characteristics of all four storage engines can be found on table~\ref{tab:storage_engines_mysql}.
\begin{table}[ht]
  \centering
  
  \begin{tabular}{p{4cm}|p{2cm}|p{2cm}|p{3cm}|p{2cm}}
    \textsc{Attribute}
  & \textsc{MyISAM}
  & \textsc{Heap}
  & \textsc{BDB}
  & \textsc{InnoDB} \\
  \hline

    \textbf{Transactions}
  & No
  & No
  & Yes
  & Yes \\

  \hline
    \textbf{Lock granularity}
  & Table
  & Table
  & Page
  & Row \\

  \hline
    \textbf{Storage}
  & Split files
  & In-memory
  & Single file per table
  & Tablespace \\

  \hline
    \textbf{Isolation levels}
  & None
  & None
  & Read committed
  & All \\

  \hline
    \textbf{Portable format}
  & Yes
  & No
  & No
  & Yes \\

  \hline
    \textbf{Referential integrity}
  & No
  & No
  & No
  & Yes \\

  \hline
    \textbf{Primary key with data}
  & No
  & No
  & Yes
  & Yes \\
    
  \hline
    \textbf{MySQL caching}
  & No
  & Yes
  & Yes
  & Yes \\

  \hline
    \textbf{Available versions}
  & All versions
  & All versions
  & MySQL-Max
  & All versions \\
  \end{tabular}
  \caption{Storage engine characteristics in MySQL}
  \label{tab:storage_engines_mysql}
\end{table}
Rails' ActiveRecord natively supports this type of database.


\subsection{PostgreSQL}
PostgreSQL is the most advanced open source database server. It was started by Michael Stonebraker at the University of California in Berkeley and had its first release in 1989. It is a DBMS that contains all the features found on other open source or commercial databases and a few more. Some of these features include~\cite{beginning_postgresql}:
\begin{itemize}
  \item Transactions;
  \item Subselects;
  \item Views;
  \item Foreign key referential integrity;
  \item Sophisticated locking;
  \item User-defined types;
  \item Inheritance;
  \item Rules;
  \item Multiple-version concurrency control;
  \item Native Microsoft Windows version;
  \item Table spaces;
  \item Ability to alter column types;
  \item  Point-in-time recovery.
\end{itemize}
PostgreSQL has some prominent users, like MySpace, who strengthen its credibility as a full-featured scalable highly-reliable relational database~\cite{petabyte_warehouses}. Rails' ActiveRecord natively supports this type of database.


\subsection{MongoDB}
MongoDB is a scalable, high performance, open source, schema-free, document-oriented database written in C++ whose first release was in early 2009. It is a combination of key-value stores, fast and highly scalable, and traditional RDBMS systems which provide structured schemas and powerful queries. Among other features, MongoDB provides~\cite{mongodb}:
\begin{itemize}
  \item Document-oriented storage (the simplicity and power of JSON-like data schemas);
  \item Dynamic queries;
  \item Full index support, extending to inner-objects and embedded arrays;
  \item Query profiling;
  \item Fast, in-place updates;
  \item Efficient storage of binary data large objects (e.g. photos and videos);
  \item Replication and fail-over support;
  \item Auto-sharding for cloud-level scalability;
  \item MapReduce for complex aggregation.
\end{itemize}
This database has been gaining popularity within the Rails community for its simplicity of use, high performance and many features that fit well within the Ruby development  philosophy~\cite{mongodb_rails}. Mongo is very performance oriented and some of its features that provide outstanding performance are~\cite{mongodb_couchdb}:
\begin{itemize}
  \item Client driver per language: native socket protocol for client/server interface (not REST);
  \item Use of memory mapped files for data storage;
  \item Collection-oriented storage (objects from the same collection are stored contiguously);
  \item Update-in-place (not MVCC);
  \item Written in C++.
\end{itemize}
Rails' ActiveRecord does not natively support MongoDB. For its usage in Rails the developer must use \textit{MongoMapper}, which provides access to Mongo database operations and natively supports Ruby objects without conversions~\cite{mongomapper}.


\section{Ruby on Rails} % (fold)
\label{state:sec:ruby_on_rails}
Rails is developed by a huge community and coordinated by its core team~\cite{rails_core_team}. However, many companies like \textit{Twitter} had the need to squeeze this frameworks' scalability and focused on some critical bottlenecks which had been identified before, like Ruby's garbage collector mentioned in Section~\ref{state:sec:ruby}.

There have been some outside performance improvements over Rails 2.3 code, though. From simple Rails' source minor patching~\cite{accunote_rails} to complete architectural makeovers~\cite{distributed_rails}, Ruby on Rails' improvements were widely attempted and some succeeded.

However, the main performance improvements targeted at Rails 2.3 can be found in Rails 3. The decoupling mentioned in Section~\ref{tech:sec:ruby_on_rails:rails3} significantly improved Rails performance and scalability, allowing faster execution times and lighter memory usage.

There were also specific performance optimizations, namely on \textit{partial} and \textit{collection} handling. Initial performance benchmarks show that partial and collection rendering and overall performance is more than two times faster~\cite{vaporware_to_awesome,rails_merb_merge_performance}. Controller-related code was also refactored to become lighter and more modular. So was \textit{ActiveRecord}, which now uses \textit{Arel}---a relational algebra framework---to generate smarter and more efficient queries before dispatching them to the database adapter.

Regarding Rails' native profiling tools, benchmarking Rails applications under Ruby 1.9 is broken, since it relies on Ruby's \textit{GC::Profiler} ``data'' method~\cite{testing_performance_old} which is missing and therefore not available~\cite{yarv_gc_profiler}.


% chapter state_of_the_art (end)

  %% State Of The Art
  \chapter{Problem Statement} % (fold)
\label{cha:problem_statement}
\headermark{Problem Statement}

\section*{} % (fold)
Ruby on Rails applications generally have scalability issues. A significant amount of applications developed using this framework exhibit this issue by publicly starting to suffer from performance and scalability issues as their popularity increases. The Ruby on Rails community is not generally focused in building highly performant systems. The lack of generic centralized information, functional and intuitive profiling tools and global awareness of this subjects's importance all contribute to this problem.

As observable on chapter~\ref{cha:state_of_the_art}, some information regarding benchmarks and configurations of the various components involved in a Rails application exists but it is, however, very scattered. Most tests and configuration evaluations are not generic, focusing on specific components, setups or purposes. It is crucial to create a solid set of conventions and guidelines that are generic and cover all components, approaching benchmarking and tweaking from a unified perspective---optimizing a Ruby on Rails application.

On a related matter, the native profiling tools in Ruby on Rails have never functioned properly when used with Ruby 1.9. While complicating the act of profiling an application, this also prevents some users from switching to this version and leveraging from its increased performance. At the same time, the profiling output formats supported are all text based. Profiling is an essential aspect of increasing an application's performance, so it is very important to fix Ruby and Rails' profiling tools, provide a seamless integration between them and the most recent versions of Ruby and add support for alternate, more intuitive profiling output formats.

As mentioned in chapter~\ref{cha:introduction}, Twitter---and a few other renowned platforms---had many scalability issues which were not discrete, some becoming quite famous. However, the global awareness of the Ruby on Rails community regarding the importance of scalability is still remarkably low. Despite having the Rails 3 API available shortly after its development began back in early 2009 and Ruby 1.9 available since 2008, most plugin developers have not added support for these versions to their plugins. Famous Rails applications, like Redmine, also lack support for Ruby 1.9 and have not started upgrading to Rails 3. Knowing the performance benefits of using the latest versions of both components, it is important to increase the community's \textit{momentum} by updating famous plugins and applications, possibly igniting an update-focused philosophy.

To create a generic guideline with centralized information, improve the current profiling tools and increase the community's sensibility on this subject it is crucial to address specific characteristics, configurations and issues of all components involved with the aforementioned perspective. 

The following sections state and justify the problems addressed in each component.

\section{Operating Systems}
Focusing on finding specific Operating System bottlenecks and solving or improving them is beyond the scope of this project. It is also unlikely to be worthwhile given the time span of the project and its wide area coverage. Furthermore, most operating systems have significant communities discussing, implementing and improving solutions to given bottlenecks.

However, there is need to determine which OS is best suited for Rails applications. The lack of generic benchmarking tools limits one's ability to fairly compare the performance of operating systems. These should be created, taking into account common tasks performed by them. 

Ascertaining which OS best handles web server load with the default configurations is also important to build a starting point for deploying and optimizing Rails applications. 

Finally, as mentioned in section~\ref{tech:sec:ruby}, Ruby is mainly developed in Linux but should easily run on many other operating systems. However, as seen on section~\ref{state:sec:operating_systems}, its performance can greatly differ depending on what OS is running it. The analysis on the performance differences of the Ruby interpreters across operating systems should continue, aside from those already determined.

\section{Ruby}
Ruby 1.9 is supposed to have a noticeably better performance than its ancestor. Most Rails applications were developed with version 1.8 and would highly benefit from the new version's improvements if they were upgraded. It is crucial to determine the benefits of this upgrade to justify the effort needed to accomplish it. 

As explained in section~\ref{state:sec:ruby}, most of the Ruby-related performance bottlenecks found in Rails applications are related to its garbage collector. Despite using a not very optimized garbage collection algorithm---\textit{mark-and-sweep}---it is not configurable, lacking the ability to adapt itself to the application it is running. As seen on section~\ref{tech:sec:ruby}, there are many benefits in having a configurable GC, so it becomes important to give YARV this flexibility. 

Finally, the integration between Rails' profiling and benchmarking tools and Ruby is outdated or, in some cases, nonexistent. It is important to update the existing tools and add support for new information and output formats, therefore improving YARV's data retrieval abilities and its integration with Rails' profiling and benchmarking facilities.

\section{Rails Web Servers}
There are many analysis concerning web server performance but unfortunately most of them are either outdated, not applicable nowadays' setups, not covering a wide range of web servers or lacking relevant data. For instance, the high quality analysis mentioned in~\ref{state:sec:rails_web_servers} does not include Unicorn, nor does it test Mongrel, Passenger and Thin behind other web servers. Furthermore, one of the critically significant but frequently missing aspect in benchmarks is memory usage.

It is also important to analyze the high-load response of each web server by stressing it to its limits and evaluating its behavior from a stability perspective. 

An up-to-date analysis covering raw performance, high-load stability and memory usage is crucial for increasing developers' awareness of the benefits and shortcomings of each web server setup. Since most web servers are natively flexible and configurable, its essential to explore and analyze their configuration options as well.

\section{Databases}
Similarly to the OS component, focusing on finding specific Database bottlenecks and solving or improving them is beyond the scope of this project. It is also unlikely to be worthwhile given the previously stated reasons.

However, Ruby plays an important role concerning databases. The current implementation of the \textit{mysql2} library converts the data between MySQL types and Ruby objects immediately after fetching each row from the database. This behavior is not optimal for situations where there are more fields being fetched than those being used. Changing the program's flow in order to trigger type casting only when the variable is actually accessed---lazy type casting---can have a significant impact on the aforementioned conditions, becoming an important issue to address.

This library is likely to replace Ruby's default in the future. It is being actively developed and provides significant ameliorations from the current library, \textit{mysql}. It is starting to gain a solid reputation within the Ruby community, posing as an excellent target for improvements concerning this component.

\section{Ruby on Rails}
The current native profiling tools for Ruby on Rails are deprecated. They do not function properly on up-to-date environments and they highly depend on Ruby's profiling abilities which are very limited. Improving Ruby's profiling tools is not enough as they need to seamlessly integrate with Rails'. For this to happen, Rails must also targeted for improvements by refactoring and improving the existent non-functional profiling tools.

Rails 3's performance is significantly better when comparing to its predecessor, as mentioned in section~\ref{state:sec:ruby_on_rails}. However, the adoption of Rails 3 will probably be difficult and lengthy since applications need to be refactored and most plugins---one of the main advantages of this framework as explained in section~\ref{tech:sec:ruby_on_rails}---need to be rewritten. To motivate its early adoption and increase this version's \textit{momentum}, some of the most famous plugins need to be ported to the new version. Getting one of the most famous applications---Redmine---to work properly on Rails 3 would also increase the awareness of the new versions' benefits. Finally, some performance bottlenecks commonly found in Rails applications for whose simple solutions exist should also addressed.

\begin{comment}
Rails scalability should be easily accessible to everyone

Improve existing tools, create/improve measurement tools

Missing performance-oriented guidelines and conventions

Missing and out-dated tools to profile Rails applications

Optimal configurations and setups
\end{comment}

  %% Problem Approach and Results
  \chapter{Problem Approach and Results} % (fold)
\label{cha:problem_approach_and_results}
\headermark{Problem Approach and Results}

Envisioning a system-wide performance optimization of a Ruby on Rails application requires targeting all the components involved from the previously mentioned centric perspective---Ruby on Rails. Conducting benchmarks, tweaking configurations, developing improved solutions and evaluating the results must be accomplished with this philosophy in mind. 

It is also necessary to perform these activities using a fair base of comparison. The most important rules used when benchmarking were:
\begin{description}
  \item[Same tests:] Use the same tests when testing component alternatives and, when impossible, keep their disparities to a minimum.
  \item[Same hardware:] Use the same hardware when testing a given component.
  \item[Similar configurations:] Use equal or, at the very least, similar configurations for all the components involved. Exceptions can be made when the purpose of the test is to benchmark the behavior with the default configurations.
\end{description}

Two distinct machines were used during this research. Due to hardware malfunction, the initial computer had to be replaced with a different one. ``Machine 1'' one was used on all the OS-related work, while ``Machine 2'' was used for all the activities related to the remaining components. Their specifications are listed on table~\ref{tab:machines_hardware_specification}.
\begin{table}[ht]
  \centering
  \caption{Hardware Specifications of the Machines in Use}
  \label{tab:machines_hardware_specification}
  
  \begin{tabular}{p{0.085\textwidth}|p{0.42\textwidth}|p{0.42\textwidth}}
    & \multicolumn{1}{c|}{\textbf{\textsc{Machine 1}}} & \multicolumn{1}{c}{\textbf{\textsc{Machine 2}}} \\ \hline

    \textbf{\textsc{CPU}} & Intel Core 2 Quad Q9300 @ 2.50GHz (FSB @ 1333MHz) & Intel Core 2 Duo ~\,E8400 @ 3.0GHz (FSB @ 1333MHz) \\ \hline
    \textbf{\textsc{RAM}} & 2x2GB DDR2 (800 MHz, Dual Channel) & 4x2GB DDR2 (800 MHz, Dual Channel) \\ \hline
    \textbf{\textsc{Hard Drive}} & Seagate ST3500620AS 500GB SATA, 16MB Cache & Seagate ST3750630AS 750GB SATA, 16MB Cache \\
  
  \end{tabular}
\end{table}

When benchmarking, the same software versions were used across all systems and environments. Table~\ref{tab:software_versions} shows the software versions that were used.
\begin{table}[ht]
  \centering
  \caption{Software Versions in Use}
  \label{tab:software_versions}
  
  \begin{tabular}{p{0.25\textwidth}|p{0.25\textwidth}}
    \multicolumn{1}{c|}{\textbf{\textsc{Software/Package}}} & \multicolumn{1}{c}{\textbf{\textsc{Version}}} \\ \hline
    
    MRI & 1.8.7 (patchlevel 249) \\ \hline
    YARV & 1.9.1 (patchlevel 378) \\ \hline
    Rails & 2.3.5 \\ \hline
    FreeBSD & 8 \\ \hline
    Linux Kernel & 2.6.26 \\ \hline
    hdparm & 9.15 \\ \hline
    gzip & 1.3.12 \\ \hline
    lame & 3.98.2 \\ \hline
    vorbis-tools & 1.2.0 \\ \hline
    Apache & 2.2.14 \\ \hline
    Nginx & 0.7.64 \\ \hline
    Cherokee & 0.99.42 \\ \hline
    Thin & 1.2.7 \\ \hline
    Unicorn & 0.97.0 \\ \hline
    Passenger & 2.2.11 \\ \hline
    autobench & 2.1.2 \\ \hline
    httperf & 0.9.0 \\ \hline
    GCC & 4.3.4 \\
        
  \end{tabular}
\end{table}
All packages were compiled from source using GCC with \textit{``-O2 -march=nocona -pipe''} as their compilation flags.

To achieve the objectives established in Chapter~\ref{cha:problem_statement}, some activities were performed.

First of all, to create the centralized set of conventions and guidelines for improving the performance of Ruby on Rails applications, there is the need to analyze and benchmark all worthy alternatives for each component.

In order to easen the process of profiling Rails applications, the native profiling tools must be fixed and upgraded. Rails and Ruby must be improved and tweaked to accommodate all changes, providing a seamless integration between each other's tools.

Finally, to improve the global awareness of this issue's importance there is the need to gather the benefits associated with it, provide a smoother transition by updating renowed Rails plugins, increase this subjects' notoriety by updating one of the most renowned Ruby on Rails projects---Redmine---and by publishing a series of articles about performance optimization on the \textit{Rails Magazine} and, ultimately, initiate a performance-oriented continuous integration for Rails by developing a benchmarking suite aimed at the framework itself. % FIXME: transition?

In order to accomplish the aforementioned tasks, every component was specifically addressed. The work done is presented and explained in the following sections.

\section{Operating Systems} % (fold)
\label{solution:sec:operating_systems}

Using Linux and BSD, the focus on this system component was to create generic benchmarking tools, determine the operating system in which common web servers perform better and to determine the OS in which the official Ruby interpreters have the best performance.

As exposed in chapter~\ref{state:sec:operating_systems}, Windows is not a suitable OS for production environments of Ruby on Rails applications because of its poor and inefficient support for this applications that use this framework, being excluded from further research. On the other hand, Mac OS X Server requires specific hardware so any comparison's would not be rigorous. Its performance is expected to be similar to BSD systems since, as mentioned in section~\ref{tech:sec:operating_systems}, its kernel is based on this OS, reducing the downside of its exclusion.

\begin{comment}
Create generic and specific tools of OS performance measurement

Find the best OS for Rails by benchmarking the most likely candidates (same hardware)

Tweak OSes configurations

OSes are already highly optimized, OS development doesn't make much sense
\end{comment}


\subsection{Development}
Concerning development, a generic benchmarking script was created. This script was based on a few commonly found tools on Unix setups and consists on 5 micro-tests and 1 macro-test, respectively:
\begin{enumerate}
  \item Use hdparm to time cached reads on the disk;
  \item Compress a 2.5GB file to ZIP format using gzip;
  \item Uncompress the previously created archive;
  \item Convert a 214MB WAV file to MP3 using lame;
  \item Convert the same 214MB WAV file to OGG using vorbis-tools;
  \item Parallelly run all the aforementioned benchmarks while extracting, compiling, installing and removing PHP 5.2.12.
\end{enumerate}
The script measures the real amount of time needed to accomplish each task, the number of voluntary context switches and the average CPU usage. GNU time is used to make all measurements except in the first test, since hdparm itself measures the amount of cached data read in 2 seconds yielding results in MB/second. All tests are ran a configurable amount of times to cancel circumstantial issues, the default being 3. Exception is made on the last test which is very heavy and lengthy, so it only runs once.

The first test aims at testing hard drive access speed, which are dependent on the filesystem in use and the OS's IO management. The second and third tests are more complex since but similar. Both read a file with considerable size from the disk, convert it and write the result. However, the main bottleneck happens when writing the result file since writing is slower process than reading and the ZIP algorithm is lightweight and fast. The fourth and fifth tests are more CPU-intensive. Audio format conversions tend to demand a significant amount of processing power. The tools in use---lame and vorbis-tools---stress the OS even further by using multiple processes and threads, inducing various context switches. Finally, the last test aims at testing the OS's ability to manage a high workload since multiple heavy tasks are being carried simultaneously, involving concurrent IO, context switches, scheduling and a few other core tasks.

\subsection{Benchmarking}
The benchmarking phase had the clear goal of defining which is the likely best OS to invest in the remaining work. It was also very important to gather data about each OS/distribution behavior so that it would be inserted in the aforementioned guidelines in conventions.

\subsubsection{Generic Benchmarking of Linux distributions}
First of all, it was important to choose one of the Linux distributions mentioned in section~\ref{tech:sec:operating_systems} to be stacked against FreeBSD, the most popular BSD distribution. A benchmark using the aforementioned generic script was performed on Ubuntu Server, Debian, CentOS and Gentoo. All distributions were running their default configurations for all packages. The results are shown in table~\ref{TABELA TABELA}.
\\
TABELA TABELA
\\
Gentoo's performance is better by a slight margin in the first test, yielding results which range from 3\% to 7\% better than the other distributions. The second test yielded similar results, with Debian's performance being very close to Gentoo's. CentOS shows serious issues in this test, being 502\% slower than Gentoo. Regarding the third test, Gentoo showed the best result, followed by Debian's. CentOS yields very poor results again, being 1903\% slower than Gentoo. Quite unexpectedly, CentOS had the best performance in the fourth test by a comfortable margin, with Debian's results being the second best once again. Ubuntu yielded the best performance in the fifth test, with Debian and Gentoo having very close results. CentOS shows the worst results by a considerable 13\% margin. Finally, Debian yielded the best results in the final test by a substantial margin. CentOS's results, once more, show a considerable performance deficiency when compared to the other distributions' results.

According to these results, Gentoo is the best distribution in CPU usage and IO operations on a single instance, present in tests 1, 2 and 3. When it comes to almost pure CPU usage, CentOS and Ubuntu yield the best results. Last but not least, Debian showed an impressive behavior handling concurrent tasks present in the last test.

CentOS's behavior was unstable and inconsistent. Given these results, it was discarded from future work.

\subsubsection{Web Server Benchmarking on Linux distributions}
Since there are still 3 possible Linux distributions to be compared with FreeBSD, a different benchmark was endured. This time, its focus was oriented towards web server performance.

This test used a simple static HTML page served by either Apache or Nginx. Using Ubuntu Server, Debian and Gentoo many requests/concurrency combinations were used, namely:
\begin{enumerate}
  \item 10000 requests, 1000 concurrent;
  \item 100000 requests, 1000 concurrent;
  \item 100000 requests, 10000 concurrent.
\end{enumerate}
Apache's ab utility was used to perform the tests. All of them were local, providing zero network overhead since the goal is to measure raw web server performance on each OS. If any request took more than 30 seconds to be replied to the test was considered a failure, as a higher response time is not acceptable in real world applications. The web server configurations were not the default ones on this test. Since some distributions loaded more modules than others and this could have a significant impact the web server performance, all unnecessary modules and options for this benchmark were removed from the configurations.

Regarding the Apache benchmark, table~\ref{WWWWWWWWWWWW} shows that Gentoo had the best performance in the first Apache test, with Debian achieving similar results. Ubuntu, however, did not cope with the other's behavior, needing a considerable amount of extra time to accomplish the same test. As seen on table~\ref{WWWWWWWWWWWWW}, Gentoo showed the best performance in the second Apache test. Debian had a considerably worse performance and Ubuntu failed this test as many requests took more than the aforementioned 30 seconds to be replied to. Finally, table~\ref{WWWWWWWWWWWWWWWW} shows the results of the last Apache benchmark, where all distributions failed to successfully complete the benchmark except Gentoo, which needed a high average amount of time to complete but was still able to reply to all requests within the established time limit.

Concerning the Nginx benchmark, table~\ref{WWWWWWWWWWWW} shows that this time it was Debian to achieve the best result on the first benchmark, yielding impressive performance. The results of the second test are shown on table~\ref{WWWWWWWWWWWW} and Debian seems to be the best performing distribution once again. The third test's results showed some unexpected results since, similarly to the Apache benchmark, neither Debian nor Ubuntu were able to cope with the high demand, leaving the best result to Gentoo which was the only distribution to successfully complete the final test. These results can be found in table~\ref{WWWWWWWWWWWW}.

Gentoo showed an excellent behavior when scaling. The difference in average time taken for each request on tests 1 and 2 of both web servers is remarkably small. It was also the only distribution to be able to cope with 100000 requests with 10000 of them concurrent, either on Apache and Nginx. These results allowed to confidently decide that Gentoo is the best distribution to compare to FreeBSD.

\subsubsection{Ruby Benchmark on Gentoo Linux and FreeBSD}
Given this research's scope, it is important to determine which of the aforementioned OSes---Gentoo Linux or FreeBSD---provide the best environment for a Ruby on Rails application. A Ruby on Rails application, as the name implies, is written in Ruby just like the framework it is using. Therefore, Ruby is a core component from Rails' perspective. The official Ruby interpreters are likely to yield different performance results on different OSes since they are mainly developed in Linux and then ported to other Operating Systems. If we take into account the already known differences stated on section~\ref{state:sec:operating_systems}, benchmarking this core component is likely to yield different results and to enable a confident assertion about the OS in which it is developed---Linux---is the best for a Ruby on Rails application or not.

For this benchmark, Antonio Cangiano's Ruby benchmarking suite~\cite[ruby-benchmarking-suite] was used. It currently contains 62 micro benchmarks which test specific Ruby features, 8 macro benchmarks which test multiple Ruby features in a single test and 3 RDoc-related benchmarks. Each benchmark ran 5 times and had a 300 second timeout. This high test variety provides a wide coverage of many Ruby features, solidly asserting about the interpreter's overall performance. All tests were ran using both Ruby interpreters, MRI (Ruby 1.8) and YARV (Ruby 1.9), in both operating systems.
\\
TABELA TABELA
\\
As seen on table~\ref{TABELA MRI}, MRI has a better overall performance in Linux. The average improvement is of 29.34\%.

Table~\ref{TABELA YARV} shows the results of the YARV benchmark. Similarly to MRI's benchmark, YARV has a better overall performance in Linux. The average improvement is 22.21\% on this test.

After eliminating FreeBSD from the benchmarking subjects, Gentoo Linux is the OS that will be used in future work. It is very stable, configurable and enables improved performance of Ruby-related software when compared to FreeBSD.

\begin{comment}
Use the generic benchmarks mentioned above

Web server benchmark (Nginx/Apache on each disto)

Ruby benchmarks on each OS

Show results and analysis
\end{comment}


\subsection{Tweaking}
There are many configurations and options that can be fine-tuned on an Operating System. Sysctl enables kernel parameter configuration at runtime. The aforeshown web server benchmarks required some optimization changes to improve the system's stability under high-load. These are shown on table~\ref{tab:sysctl}.
\begin{table}[ht]
  \centering
  
  \begin{tabular}{p{0.05\textwidth}|p{0.32\textwidth}|p{0.25\textwidth}}
    \textsc{\#} 
  & \textsc{Name}
  & \textsc{Value} \\
  \hline
  1 & net.core.rmem\_max & 16777216 \\
  2 & net.core.wmem\_max & 16777216 \\
  3 & net.ipv4.tcp\_rmem & 4096~\, 87380~\, 16777216 \\  
  4 & net.ipv4.tcp\_wmem & 4096~\, 87380~\, 16777216 \\
  5 & net.core.netdev\_max\_backlog & 4096 \\
  6 & net.core.somaxconn & 4096 \\
  7 & net.ipv4.tcp\_tw\_reuse & 1 \\
  8 & net.ipv4.tcp\_tw\_recycle & 1 \\
  9 & net.ipv4.tcp\_fin\_timeout & 15 \\
  10 & net.ipv4.tcp\_timestamps & 0 \\
  11 & net.ipv4.tcp\_orphan\_retries & 1 \\
  
  \end{tabular}
  \caption{Sysctl options and values}
  \label{tab:sysctl}
\end{table}
Options 1, 2, 3 and 4 increase the TCP buffers on read/write, improving the system performance when dealing with big transfers. Options 5 and 6 increase the number of connections which are allowed to be queued behind a busy kernel. Options 7 and 8 enable socket reusing and fast socket recycling. Option 9 decreases the time allowed for a socket to exists without a connection. Option 10 disables timestamps in packet headers, reducing the packet's size. Finally, option 11 decreases the number of retries before killing the TCP connection.

The number of opened files limit also had to be increased in the system's limits configuration. It defaults to 1024 which is very low on a server, taking into account that each socket connection uses a file on a UNIX system. This would generally cap the system's concurrency ability to ~1000, so it was increased to 65536.

A few other options are worth investigating. Many server-oriented distributions use the Deadline IO scheduler which gives a higher priority to read requests, while others use the CFQ scheduler which is commonly found on desktop systems. Preemption should also be disabled on a server kernel. In non-preemptive configurations, kernel code runs until completion---the scheduler can't touch it until it's finished. Server kernels should also have their timer interrupt rate set to 100Hz, which causes higher latency but lower overhead, yielding superior raw processing power.

On a side note, all the aforementioned configuration changes were in use in all benchmarks.

\section{Ruby} % (fold)
\label{solution:sec:ruby}

\subsection{Methodology}
The main focus was on the latest version of the official Ruby interpreter, being divided into three core activities. First of all, determining the real benefits of upgrading to the latest Ruby version, 1.9. If the upgrade was proven worthy, focus would shift towards improving YARV's GC by increasing its flexibility. After that, Ruby's profiling and information retrieval capabilities were enhanced.

\begin{comment}
Determine the benefits of upgrading to the latest version, 1.9

Improve ruby 1.9's GC (main weakpoint)

Add GC configurability

Improve Ruby 1.9's profiling abilities and information retrieval
\end{comment}


\subsection{Development}
Port escolinhas to 1.9

Add GC configurability

Add the ability to store and retrieve profiling information from ruby

Ported a graphical profiling tool that enables call stacks w/ time


\subsection{Benchmarking}
Benchmark different ruby versions (and my tweaked ruby version) and analysis


\section{Rails Web Servers} % (fold)
\label{solution:sec:rails_web_servers}

\subsection{Methodology}
The efforts concerning Rails web servers were centered in determining which is best for each situation. Memory usage and stability are important factors and were taken into account when evaluating each alternative. Thin, Unicorn and Passenger were used in the aforementioned tests.



%Benchmark web servers, determine which is best for each situation, memory usage and stability

\subsection{Benchmarking}
Show all the benchmarks, the simple and the complex ones + analysis

\subsection{Tweaking}
Explore configurations and architectures


\section{Databases} % (fold)
\label{solution:sec:databases}

\subsection{Methodology}
Adding lazy type conversion to the alternative library mentioned in section~\ref{tech:sec:databases}---\textit{mysql2}---was the main focus concerning databases. However, exploring MySQL's native caching solutions was also addressed.

%TODO: falar de que é muito comum?

%Don't invest in the DB directly - not connected to Ruby. Approach adapters instead.

\subsection{Tweaking}
Native mysql caching at escolinhas

\subsection{Development}
mysql2 with lazy type convertion

\section{Ruby on Rails} % (fold)
\label{solution:sec:ruby_on_rails}

\subsection{Methodology}
The main focus regarding Ruby on Rails itself was to improve the current tools for profiling and benchmarking applications. Motivating the adoption of the soon to be released version of this framework---Rails 3---by the community was also a goal regarding this component. Since the current version of Rails is still 2.3, minor tweaks were explored to improve this versions' performance.

\begin{comment}
Create tools to improve Rails

Push the Rails community forward
\end{comment}

\subsection{Tweaking}
Blog stuff

\subsection{Benchmarking}
Blog stuff

\subsection{Development}
Refactored profiling facilities (which were broken)

Ported many plugins to Rails 3 (Escolinhas uses them)

Porting Rails' most famous application to Rails 3

Benchmarking CI




  %% Conclusions
  \chapter{Conclusions} % (fold)
\label{cha:conclusions}
\headermark{Conclusions}
\section*{} % (fold)

\begin{comment}
This is deprecated!!!!!!

The existing research provides a solid base for future work on this subject. The endured technology overview allowed the creation of an expectancy base for each component. The state of the art provided an overview of the current engaged work on related, if not similar, subjects.

All the gathered and structured information permitted option narrowing when it comes to future work. Taking into account the research and conclusions of some related work, some components will be left out of future benchmarking, tuning, testing and improvement.

On the Operating System's concern, Microsoft's operating system will be left out of future efforts since its Ruby and Rails support is significantly poor and inefficient. The lack of benchmark and testing information about BSD and the fact that its performance should be similar to Linux's include it in the test group. Concluding, the Operating Systems contempt by this project will be Linux and BSD.

Ruby is currently in version 1.9 which has many improvements over the previous version, both performance and not performance-oriented. Since YARV is the only one to support its specification to full extent, this will be the only interpreter targeted for improvements. Luckily, Rubinius' 1.9 support is growing day by day and it is close to 100\%. Only time will tell if this interpreter will be considered as well.

Rails web servers' situation is more flat. Only WEBrick is left out for obvious reasons --- it was Rails' pioneer web server but lacks the efficiency to compete with nowadays alternatives.

When it comes to relational databases, MySQL's scaling and performance capabilities shine. On the other hand, MongoDB presents itself as a worthy alternative but it has an important characteristic --- it is a schema-less database. PostgreSQL, although providing many interesting features, is not efficient enough to compete with MySQL, being left out of the test group for further work. MySQL and MongoDB are targeted for all the aforementioned improvement cycle.

Rails is growing fast and version 3 will soon to be publicly released. It solves many existing performance bottlenecks in Rails 2.3 and improves its predecessor code by a great deal. Working on Rails 2.3 would not be very productive since it will soon become deprecated and inefficient when compared with the most recent version. Version 3 will be the main target focus for further development.

The application itself, \textit{Escolinhas}, would benefit from huge performance gains just by being ported to Ruby 1.9 and Rails 3. The remaining improvements are dependant on the improvement cycles of all the components mentioned before, as performance bottlenecks must be found and solved on each setup.

Ultimately this research provided a careful overview over the involved technologies and their current state, sustaining the creation of a solid research and development plan for future efforts.
\end{comment}

\section{Conclusions}
Guidelines and conventions are created

Escolinhas is more scalable

Open-source contribution with considerable range and big impact


\section{Summary of Contributions}

Created generic guidelines and conventions

Improved Rails profiling abilities

Improved Ruby's profiling abilities

Improved Ruby's GC flexibility

Ported Escolinhas to Ruby 1.9

Ported Redmine to Ruby 1.9 and Rails 3

Ported the plugins used by Escolinhas to Rails 3

Demonstrated Ruby's performance on different OSes

Demonstrated OSes performance with generic and web-oriented benchmarks

Creation of web server benchmarking scripts

In-depth analysis of the current web servers performance

Demonstrated the performance of the latest Ruby against the most used version

Joint of many micro-researches (?)


\section{Future Research}

Research other databases aside mysql

Invest in alternative Ruby interpreters (which should be stable in a near future)

Develop a generational GC for Ruby


  
  
  %%----------------------------------------
  %% Final materials
  %%----------------------------------------

  \begin{singlespace}
    \PrintBib{references}

    %% Index
    %% Uncomment next command if index is required,
    %% don't forget to run ``makeindex mieic'' command
    %\PrintIndex

    %% Comment next 2 commands if numbered appendixes not used
    \appendix
    \chapter{Ruby 1.9 Encoding Patch} % (fold)
\label{ap:ruby19_encoding_patch}

The Ruby patch developed to fix the encoding issues when using non-ASCII characters on version 2.3.5 of Ruby on Rails.\\\\

\begin{lstlisting}[language=ruby]
# coding: UTF-8

# TODO: Most of these issues are not present in Rails 3. Remove this when updating.

# Force mysql rows to be UTF-8 (see rails.lighthouseapp.com/projects/8994/tickets/2476)
require 'mysql'
 
class Mysql::Result
  def encode(value, encoding = "utf-8")
    (String  === value && value.respond_to?(:force_encoding)) ? value.force_encoding(encoding) : value
  end
  
  def each_utf8(&block)
    each_orig do |row|
      yield row.map {|col| encode(col) }
    end
  end
  alias each_orig each
  alias each each_utf8
 
  def each_hash_utf8(&block)
    each_hash_orig do |row|
      row.each {|k, v| row[k] = encode(v) }
      yield(row)
    end
  end
  alias each_hash_orig each_hash
  alias each_hash each_hash_utf8
end

# fix template rendering
module ActionView
  # NOTE: The template that this mixin is being included into is frozen
  # so you cannot set or modify any instance variables
  module Renderable #:nodoc:
    extend ActiveSupport::Memoizable


    private    
    def compile!(render_symbol, local_assigns)
        locals_code = local_assigns.keys.map { |key| "#{key} = local_assigns[:#{key}];" }.join

        source = <<-end_src
          def #{render_symbol}(local_assigns)
            old_output_buffer = output_buffer;#{locals_code};#{compiled_source}
          ensure
            self.output_buffer = old_output_buffer
          end
        end_src
        
        # Workaround for erb
        source.force_encoding('utf-8') if '1.9'.respond_to?(:force_encoding)

        begin
          ActionView::Base::CompiledTemplates.module_eval(source, filename, 0)
        rescue Errno::ENOENT => e
          raise e # Missing template file, re-raise for Base to rescue
        rescue Exception => e # errors from template code
          if logger = defined?(ActionController) && Base.logger
            logger.debug "ERROR: compiling #{render_symbol} RAISED #{e}"
            logger.debug "Function body: #{source}"
            logger.debug "Backtrace: #{e.backtrace.join("\n")}"
          end

          raise ActionView::TemplateError.new(self, {}, e)
        end
      end

  end
end

# the previous fix causes issues in uploaded files encoding, fixed here
module ActionController
  class Request
    private

      # Convert nested Hashs to HashWithIndifferentAccess and replace
      # file upload hashs with UploadedFile objects
      def normalize_parameters(value)
        case value
        when Hash
          if value.has_key?(:tempfile)
            upload = value[:tempfile]
            upload.extend(UploadedFile)
            upload.original_path = value[:filename]
            upload.content_type = value[:type]
            upload
          else
            h = {}
            value.each { |k, v| h[k] = normalize_parameters(v) }
            h.with_indifferent_access
          end
        when Array
          value.map { |e| normalize_parameters(e) }
        else
          value.force_encoding(Encoding::UTF_8) if value.respond_to?(:force_encoding)
          value
        end
      end
  end
end
\end{lstlisting}

    \chapter{Ruby 1.9 Encoding Task} % (fold)
\label{ap:ruby19_encoding_task}

The rake task developed to automatically manage all Ruby files and set their default encoding to UTF-8. It inserts the encoding header when needed and standardizes the variant in use to ``\# coding: UTF-8''.\\\\

\begin{lstlisting}[language=ruby]
desc "Manage the encoding header of Ruby files"
task :check_encoding_headers => :environment do
  files = Array.new
  ["*.rb", "*.rake"].each do |extension|
    files.concat(Dir[ File.join(Dir.getwd.split(/\\/), "**", extension) ])
  end

  files.each do |file|
    content = File.read(file)
    next if content[0..16] == "# coding: UTF-8\n\n"
    
    ["\n\n", "\n"].each do |file_end|
      content = content.gsub(/(# encoding: UTF-8#{file_end})|(# coding: UTF-8#{file_end})|(# -*- coding: UTF-8 -*-#{file_end})/i, "")
    end

    new_file = File.open(file, "w")
    new_file.write("# coding: UTF-8\n\n"+content)
    new_file.close
  end
end

\end{lstlisting}

    \chapter{Ruby 1.9 Configuration} % (fold)
\label{ap:ruby19_configuration}

The following command snippet starts the Ruby interpreter with an example configuration on UNIX.\\\\

\begin{lstlisting}[language=bash]
export RUBY_HEAP_SLOTS_INCREMENT=500000
export RUBY_HEAP_MIN_SLOTS=500000
export RUBY_HEAP_SLOTS_GROWTH_FACTOR=1.1
export RUBY_GC_MALLOC_LIMIT=40000000
export RUBY_HEAP_FREE_MIN=100000
ruby
\end{lstlisting}

    \chapter{Ruby-prof HTML Stack Printer} % (fold)
\label{ap:ruby-prof_html_stack_printer}

A snippet of the HTML output of this graphic hierarchical printer. It contains many information like the relative time elapsed of each call, the number of times a given call occured and even background colors to highlight faster and slower calls, among others.\\\\

\begin{figure}[h]
  \centering
    \includegraphics[width=0.9\textwidth]{ruby-prof_html_stack_printer}
  \caption{Ruby-prof HTML Stack Printer}
  \label{fig:ruby-prof_html_stack_printer}
\end{figure}

    \chapter{Memory Usage Monitor Script} % (fold)
\label{ap:ruby19_encoding_patch}

A small Ruby script that measures the memory usage of specified processes. The user can configure the refresh interval and the output directory. The result is recorded in CSV.\\\\

\begin{lstlisting}
logs = Hash.new
temp_dir = ARGV[0]
delta = ARGV[1].to_f != 0 ? ARGV[1] : 1

while true do
  ARGV[1..-1].each do |arg|
    next if ARGV[1].to_f != 0
    (`pidof #{arg}`.split-logs.keys).each do |pid|
      logs[pid] = File.open("#{temp_dir}/#{(arg.gsub(/[^a-zA-Z 0-9]/, "")).gsub(/\s/,'-')}.#{pid}.csv", "w")
    end
  end

  logs.each do |pid, log|
    begin
      f = File.open("/proc/#{pid}/status", 'r').read
    rescue
      next
    end
    
    begin
      vmpeak = f.match('VmPeak:\s+(\d+)\s+kB')[1]
      vmsize = f.match('VmSize:\s+(\d+)\s+kB')[1]
      vmrss = f.match('VmRSS:\s+(\d+)\s+kB')[1]

      log.write("#{vmpeak},#{vmsize},#{vmrss}\n")
      log.flush
    rescue
      log.close
      logs.delete(pid)
    end
  end
  
  sleep delta.to_i
end
\end{lstlisting}

    \chapter{Lazy Type Casting in mysql2} % (fold)
\label{ap:mysql2_patch}

A patch which enables lazy type casting of fields on of Ruby's \textit{MySQL} database libraries, \textit{mysql2}.

\begin{lstlisting}
diff --git a/ext/mysql2_ext.c b/ext/mysql2_ext.c
index 3f791d3..fadea7b 100644
--- a/ext/mysql2_ext.c
+++ b/ext/mysql2_ext.c
@@ -370,7 +370,7 @@ static VALUE nogvl_read_query_result(void *ptr)
   return res == 0 ? Qtrue : Qfalse;
 }
 
-/* mysql_store_result may (unlikely) read rows off the socket */
+/*  may (unlikely) read rows off the socket */
 static VALUE nogvl_store_result(void *ptr)
 {
   MYSQL * client = ptr;
@@ -521,92 +521,8 @@ static VALUE rb_mysql_result_fetch_row(int argc, VALUE * argv, VALUE self) {
       }
       rb_ary_store(wrapper->fields, i, field);
     }
-
+    VALUE val = INT2NUM(wrapper->lastRowProcessed+i);
     if (row[i]) {
-      VALUE val;
-      switch(fields[i].type) {
-        case MYSQL_TYPE_NULL:       // NULL-type field
-          val = Qnil;
-          break;
-        case MYSQL_TYPE_BIT:        // BIT field (MySQL 5.0.3 and up)
-          val = rb_str_new(row[i], fieldLengths[i]);
-          break;
-        case MYSQL_TYPE_TINY:       // TINYINT field
-        case MYSQL_TYPE_SHORT:      // SMALLINT field
-        case MYSQL_TYPE_LONG:       // INTEGER field
-        case MYSQL_TYPE_INT24:      // MEDIUMINT field
-        case MYSQL_TYPE_LONGLONG:   // BIGINT field
-        case MYSQL_TYPE_YEAR:       // YEAR field
-          val = rb_cstr2inum(row[i], 10);
-          break;
-        case MYSQL_TYPE_DECIMAL:    // DECIMAL or NUMERIC field
-        case MYSQL_TYPE_NEWDECIMAL: // Precision math DECIMAL or NUMERIC field (MySQL 5.0.3 and up)
-          val = rb_funcall(cBigDecimal, intern_new, 1, rb_str_new(row[i], fieldLengths[i]));
-          break;
-        case MYSQL_TYPE_FLOAT:      // FLOAT field
-        case MYSQL_TYPE_DOUBLE:     // DOUBLE or REAL field
-          val = rb_float_new(strtod(row[i], NULL));
-          break;
-        case MYSQL_TYPE_TIME: {     // TIME field
-          int hour, min, sec, tokens;
-          tokens = sscanf(row[i], "%2d:%2d:%2d", &hour, &min, &sec);
-          val = rb_funcall(rb_cTime, intern_local, 6, INT2NUM(0), INT2NUM(1), INT2NUM(1), INT2NUM(hour), INT2NUM(min), INT2NUM(sec));
-          break;
-        }
-        case MYSQL_TYPE_TIMESTAMP:  // TIMESTAMP field
-        case MYSQL_TYPE_DATETIME: { // DATETIME field
-          int year, month, day, hour, min, sec, tokens;
-          tokens = sscanf(row[i], "%4d-%2d-%2d %2d:%2d:%2d", &year, &month, &day, &hour, &min, &sec);
-          if (year+month+day+hour+min+sec == 0) {
-            val = Qnil;
-          } else {
-            if (month < 1 || day < 1) {
-              rb_raise(cMysql2Error, "Invalid date: %s", row[i]);
-              val = Qnil;
-            } else {
-              val = rb_funcall(rb_cTime, intern_local, 6, INT2NUM(year), INT2NUM(month), INT2NUM(day), INT2NUM(hour), INT2NUM(min), INT2NUM(sec));
-            }
-          }
-          break;
-        }
-        case MYSQL_TYPE_DATE:       // DATE field
-        case MYSQL_TYPE_NEWDATE: {  // Newer const used > 5.0
-          int year, month, day, tokens;
-          tokens = sscanf(row[i], "%4d-%2d-%2d", &year, &month, &day);
-          if (year+month+day == 0) {
-            val = Qnil;
-          } else {
-            if (month < 1 || day < 1) {
-              rb_raise(cMysql2Error, "Invalid date: %s", row[i]);
-              val = Qnil;
-            } else {
-              val = rb_funcall(cDate, intern_new, 3, INT2NUM(year), INT2NUM(month), INT2NUM(day));
-            }
-          }
-          break;
-        }
-        case MYSQL_TYPE_TINY_BLOB:
-        case MYSQL_TYPE_MEDIUM_BLOB:
-        case MYSQL_TYPE_LONG_BLOB:
-        case MYSQL_TYPE_BLOB:
-        case MYSQL_TYPE_VAR_STRING:
-        case MYSQL_TYPE_VARCHAR:
-        case MYSQL_TYPE_STRING:     // CHAR or BINARY field
-        case MYSQL_TYPE_SET:        // SET field
-        case MYSQL_TYPE_ENUM:       // ENUM field
-        case MYSQL_TYPE_GEOMETRY:   // Spatial fielda
-        default:
-          val = rb_str_new(row[i], fieldLengths[i]);
-#ifdef HAVE_RUBY_ENCODING_H
-          // rudimentary check for binary content
-          if ((fields[i].flags & BINARY_FLAG) || fields[i].charsetnr == 63) {
-            rb_enc_associate_index(val, binaryEncoding);
-          } else {
-            rb_enc_associate_index(val, utf8Encoding);
-          }
-#endif
-          break;
-      }
       rb_hash_aset(rowHash, field, val);
     } else {
       rb_hash_aset(rowHash, field, Qnil);
@@ -651,20 +567,20 @@ static VALUE rb_mysql_result_each(int argc, VALUE * argv, VALUE self) {
         wrapper->lastRowProcessed++;
       }
 
-      if (row == Qnil) {
+      /*if (row == Qnil) {
         // we don't need the mysql C dataset around anymore, peace it
         rb_mysql_result_free_result(wrapper);
         return Qnil;
-      }
+      }*/
 
       if (block != Qnil) {
         rb_yield(row);
       }
     }
-    if (wrapper->lastRowProcessed == wrapper->numberOfRows) {
+    /*if (wrapper->lastRowProcessed == wrapper->numberOfRows) {
       // we don't need the mysql C dataset around anymore, peace it
       rb_mysql_result_free_result(wrapper);
-    }
+    }*/
   }
 
   return wrapper->rows;
@@ -686,6 +602,118 @@ static VALUE rb_raise_mysql2_error(MYSQL *client) {
   return Qnil;
 }
 
+static VALUE rb_mysql_result_cast(VALUE self, VALUE index) {
+  mysql2_result_wrapper * wrapper;
+  MYSQL_FIELD * field = NULL;
+  MYSQL_ROW row;
+  VALUE val;
+  unsigned long * fieldLengths;
+  void * ptr;
+  
+  GetMysql2Result(self, wrapper);
+  
+  if (wrapper->numberOfFields == 0) {
+    wrapper->numberOfFields = mysql_num_fields(wrapper->result);
+    wrapper->fields = rb_ary_new2(wrapper->numberOfFields);
+  }
+  
+  unsigned long r = index / wrapper->numberOfRows;
+  int c = index % wrapper->numberOfFields;
+  
+  ptr = wrapper->result;
+  mysql_data_seek(ptr, r); // <--- Segmentation fault
+  row = (MYSQL_ROW)rb_thread_blocking_region(nogvl_fetch_row, ptr, RUBY_UBF_IO, 0);
+  
+  field = mysql_fetch_field_direct(ptr, c);
+  fieldLengths = mysql_fetch_lengths(wrapper->result);
+  
+  switch(field->type) {
+    case MYSQL_TYPE_NULL:       // NULL-type field
+      val = Qnil;
+      break;
+    case MYSQL_TYPE_BIT:        // BIT field (MySQL 5.0.3 and up)
+      val = rb_str_new(row[c], fieldLengths[c]);
+      break;
+    case MYSQL_TYPE_TINY:       // TINYINT field
+    case MYSQL_TYPE_SHORT:      // SMALLINT field
+    case MYSQL_TYPE_LONG:       // INTEGER field
+    case MYSQL_TYPE_INT24:      // MEDIUMINT field
+    case MYSQL_TYPE_LONGLONG:   // BIGINT field
+    case MYSQL_TYPE_YEAR:       // YEAR field
+      val = rb_cstr2inum(row[c], 10);
+      break;
+    case MYSQL_TYPE_DECIMAL:    // DECIMAL or NUMERIC field
+    case MYSQL_TYPE_NEWDECIMAL: // Precision math DECIMAL or NUMERIC field (MySQL 5.0.3 and up)
+      val = rb_funcall(cBigDecimal, intern_new, 1, rb_str_new(row[c], fieldLengths[c]));
+      break;
+    case MYSQL_TYPE_FLOAT:      // FLOAT field
+    case MYSQL_TYPE_DOUBLE:     // DOUBLE or REAL field
+      val = rb_float_new(strtod(row[c], NULL));
+      break;
+    case MYSQL_TYPE_TIME: {     // TIME field
+      int hour, min, sec, tokens;
+      tokens = sscanf(row[c], "%2d:%2d:%2d", &hour, &min, &sec);
+      val = rb_funcall(rb_cTime, intern_local, 6, INT2NUM(0), INT2NUM(1), INT2NUM(1), INT2NUM(hour), INT2NUM(min), INT2NUM(sec));
+      break;
+    }
+    case MYSQL_TYPE_TIMESTAMP:  // TIMESTAMP field
+    case MYSQL_TYPE_DATETIME: { // DATETIME field
+      int year, month, day, hour, min, sec, tokens;
+      tokens = sscanf(row[c], "%4d-%2d-%2d %2d:%2d:%2d", &year, &month, &day, &hour, &min, &sec);
+      if (year+month+day+hour+min+sec == 0) {
+        val = Qnil;
+      } else {
+        if (month < 1 || day < 1) {
+          rb_raise(cMysql2Error, "Invalid date: %s", row[c]);
+          val = Qnil;
+        } else {
+          val = rb_funcall(rb_cTime, intern_local, 6, INT2NUM(year), INT2NUM(month), INT2NUM(day), INT2NUM(hour), INT2NUM(min), INT2NUM(sec));
+        }
+      }
+      break;
+    }
+    case MYSQL_TYPE_DATE:       // DATE field
+    case MYSQL_TYPE_NEWDATE: {  // Newer const used > 5.0
+      int year, month, day, tokens;
+      tokens = sscanf(row[c], "%4d-%2d-%2d", &year, &month, &day);
+      if (year+month+day == 0) {
+        val = Qnil;
+      } else {
+        if (month < 1 || day < 1) {
+          rb_raise(cMysql2Error, "Invalid date: %s", row[c]);
+          val = Qnil;
+        } else {
+          val = rb_funcall(cDate, intern_new, 3, INT2NUM(year), INT2NUM(month), INT2NUM(day));
+        }
+      }
+      break;
+    }
+    case MYSQL_TYPE_TINY_BLOB:
+    case MYSQL_TYPE_MEDIUM_BLOB:
+    case MYSQL_TYPE_LONG_BLOB:
+    case MYSQL_TYPE_BLOB:
+    case MYSQL_TYPE_VAR_STRING:
+    case MYSQL_TYPE_VARCHAR:
+    case MYSQL_TYPE_STRING:     // CHAR or BINARY field
+    case MYSQL_TYPE_SET:        // SET field
+    case MYSQL_TYPE_ENUM:       // ENUM field
+    case MYSQL_TYPE_GEOMETRY:   // Spatial fielda
+    default:
+      val = rb_str_new(row[c], fieldLengths[c]);
+#ifdef HAVE_RUBY_ENCODING_H
+      // rudimentary check for binary content
+      if ((field->flags & BINARY_FLAG) || field->charsetnr == 63) {
+        rb_enc_associate_index(val, binaryEncoding);
+      } else {
+        rb_enc_associate_index(val, utf8Encoding);
+      }
+#endif
+      break;
+  }
+  
+  return val;
+}
+
 /* Ruby Extension initializer */
 void Init_mysql2_ext() {
   rb_require("date");
@@ -709,6 +737,7 @@ void Init_mysql2_ext() {
   rb_define_method(cMysql2Client, "async_result", rb_mysql_client_async_result, 0);
   rb_define_method(cMysql2Client, "last_id", rb_mysql_client_last_id, 0);
   rb_define_method(cMysql2Client, "affected_rows", rb_mysql_client_affected_rows, 0);
+  rb_define_method(cMysql2Client, "cast", rb_mysql_result_cast, 1);
 
   cMysql2Error = rb_define_class_under(mMysql2, "Error", rb_eStandardError);
   rb_define_method(cMysql2Error, "error_number", rb_mysql_error_error_number, 0);
@@ -716,6 +745,9 @@ void Init_mysql2_ext() {
 
   cMysql2Result = rb_define_class_under(mMysql2, "Result", rb_cObject);
   rb_define_method(cMysql2Result, "each", rb_mysql_result_each, -1);
+  
+  //cMysql2Type = rb_define_class_under(mMysql2, "Type", rb_cObject);
+  //rb_define_singleton_method(cMysql2Type, "cast", rb_mysql_result_cast, 1);
 
   VALUE mEnumerable = rb_const_get(rb_cObject, rb_intern("Enumerable"));
   rb_include_module(cMysql2Result, mEnumerable);
diff --git a/lib/active_record/connection_adapters/mysql2_adapter.rb b/lib/active_record/connection_adapters/mysql2_adapter.rb
index 825dd44..7daa537 100644
--- a/lib/active_record/connection_adapters/mysql2_adapter.rb
+++ b/lib/active_record/connection_adapters/mysql2_adapter.rb
@@ -14,6 +14,12 @@ module ActiveRecord
   module ConnectionAdapters
     class Mysql2Column < Column
       BOOL = "tinyint(1)".freeze
+      
+      def initialize(name, default, connection, sql_type = nil, null = true)
+        @connection = connection
+        super(name, default, sql_type, null)
+      end
+      
       def extract_default(default)
         if sql_type =~ /blob/i || type == :text
           if default.blank?
@@ -48,43 +54,47 @@ module ActiveRecord
           when :boolean       then Object
         end
       end
-
+      
+      # Gets the value (from the index)
       def type_cast(value)
         return nil if value.nil?
         case type
-          when :string                then value
-          when :text                  then value
-          when :integer               then value.to_i rescue value ? 1 : 0
-          when :float                 then value.to_f # returns self if it's already a Float
-          when :decimal               then self.class.value_to_decimal(value)
-          when :datetime, :timestamp  then value.class == Time ? value : self.class.string_to_time(value)
-          when :time                  then value.class == Time ? value : self.class.string_to_dummy_time(value)
-          when :date                  then value.class == Date ? value : self.class.string_to_date(value)
-          when :binary                then value
-          when :boolean               then self.class.value_to_boolean(value)
-          else value
+          when :string    then Mysql2::Type.cast(value)
+          when :text      then Mysql2::Type.cast(value)
+          when :integer   then Mysql2::Type.cast(value).to_i rescue Mysql2::Type.cast(value) ? 1 : 0
+          when :float     then Mysql2::Type.cast(value).to_f
+          when :decimal   then self.class.value_to_decimal(Mysql2::Type.cast(value))
+          when :datetime  then self.class.string_to_time(Mysql2::Type.cast(value))
+          when :timestamp then self.class.string_to_time(Mysql2::Type.cast(value))
+          when :time      then self.class.string_to_dummy_time(Mysql2::Type.cast(value))
+          when :date      then self.class.string_to_date(Mysql2::Type.cast(value))
+          when :binary    then self.class.binary_to_string(Mysql2::Type.cast(value))
+          when :boolean   then self.class.value_to_boolean(Mysql2::Type.cast(value))
+          else Mysql2::Type.cast(value)
         end
       end
 
       def type_cast_code(var_name)
         case type
-          when :string                then nil
-          when :text                  then nil
-          when :integer               then "#{var_name}.to_i rescue #{var_name} ? 1 : 0"
-          when :float                 then "#{var_name}.to_f"
-          when :decimal               then "#{self.class.name}.value_to_decimal(#{var_name})"
-          when :datetime, :timestamp  then "#{var_name}.class == Time ? #{var_name} : #{self.class.name}.string_to_time(#{var_name})"
-          when :time                  then "#{var_name}.class == Time ? #{var_name} : #{self.class.name}.string_to_dummy_time(#{var_name})"
-          when :date                  then "#{var_name}.class == Date ? #{var_name} : #{self.class.name}.string_to_date(#{var_name})"
-          when :binary                then nil
-          when :boolean               then "#{self.class.name}.value_to_boolean(#{var_name})"
+          when :string    then "Mysql2::Type.cast(#{var_name})"
+          when :text      then "Mysql2::Type.cast(#{var_name})"
+          when :integer   then "(Mysql2::Type.cast(#{var_name}).to_i rescue Mysql2::Type.cast(#{var_name}) ? 1 : 0)"
+          when :float     then "Mysql2::Type.cast(#{var_name}).to_f"
+          when :decimal   then "#{self.class.name}.value_to_decimal(Mysql2::Type.cast(#{var_name}))"
+          when :datetime  then "#{self.class.name}.string_to_time(Mysql2::Type.cast(#{var_name}))"
+          when :timestamp then "#{self.class.name}.string_to_time(Mysql2::Type.cast(#{var_name}))"
+          when :time      then "#{self.class.name}.string_to_dummy_time(Mysql2::Type.cast(#{var_name}))"
+          when :date      then "#{self.class.name}.string_to_date(Mysql2::Type.cast(#{var_name}))"
+          when :binary    then "#{self.class.name}.binary_to_string(Mysql2::Type.cast(#{var_name}))"
+          when :boolean   then "#{self.class.name}.value_to_boolean(Mysql2::Type.cast(#{var_name}))"
           else nil
         end
       end
 
       private
         def simplified_type(field_type)
-          return :boolean if Mysql2Adapter.emulate_booleans && field_type.downcase.index(BOOL)
+          puts field_type
+          return :boolean if Mysql2Adapter.emulate_booleans && @connection.cast(field_type).downcase.index(BOOL)
           return :string  if field_type =~ /enum/i
           return :integer if field_type =~ /year/i
           super
@@ -414,7 +424,7 @@ module ActiveRecord
         columns = []
         result = execute(sql, :skip_logging)
         result.each(:symbolize_keys => true) { |field|
-          columns << Mysql2Column.new(field[:Field], field[:Default], field[:Type], field[:Null] == "YES")
+          columns << Mysql2Column.new(field[:Field], field[:Default], @connection, field[:Type], field[:Null] == "YES")
         }
         columns
       end
@@ -592,4 +602,4 @@ module ActiveRecord
         end
     end
   end
-end
\ No newline at end of file
+end
\end{lstlisting}

    \chapter{“Scaling Rails” Article on Rails Magazine} % (fold)
\label{ap:rails_magazine}

Scaling Rails\\
\textit{by Gonçalo Silva}\\\\\\
The term Web 2.0, born somewhere in 2001, is related to improving the first version of the Web. It aims at improving user experiences, by providing better usability and more dynamic content. Many web frameworks, including Ruby on Rails, were born as part of this huge web momentum that we still live in nowadays.\\\\
Most websites are built to provide great user experiences. Recent studies show that users won't wait longer than 8 seconds before leaving a slow or unresponsive website. This value keeps getting lower and lower as users become more demanding.\\\\
This is the introductory article in a short series related to Ruby on Rails Scalability and Performance Optimization. This great framework's performance is conditioned by every related component, besides itself.

\section{Website Performance}
Developers often oversee that web applications need to be fast and very responsive as part of a richer user experience. Having a highly efficient platform will generally allow lower expenses on hardware but also lower response times, making its users happier. In some cases this  need can be extreme---a high-demand platform strives for scalability as its users keep growing.\\\\
Ruby on Rails is widely known for being optimized for programmer productivity and happiness, but its scalability or performance are not generally favored. Many well-known platforms like Twitter or Scribd have put enormous efforts in improving these characteristics and sometimes faced a few issues while doing it---we all know the famous ``fail whale''.

\section{System Resources}
Very little people have access to topnotch resources. Most need to deploy and maintain a Rails application on a shared host, limited VPS or even a dedicated server. High-end computers or server clusters are not easily accessible by the masses but every developer should to be able to provide great services with limited resources.\\\\
The answer to every scalability issue is not ``just throw hardware at it''.

\section{Involved Components}
Most servers run Linux, while a few use FreeBSD. Every operating system has a different philosophy to almost everything – from the file system to networking I/O, and these details can impact all the other components.\\\\
Some applications use Ruby 1.8, while others are already riding Ruby 1.9. There are many Ruby interpreters, from MRI to YARV, including widely-known implementations like Ruby Enterprise Edition, JRuby or Rubinius. Each of these has its particular characteristics, offering different advantages and disadvantages.\\\\
Very few applications are built or have been ported to Rails 3, which suffered huge improvements on performance. Porting is an important step as Rails 3 provides reduced computing times and memory usage, when compared to its predecessor.\\\\
When it comes to web servers, the number of choices are tremendous. From the old combination of Apache and Mongrel, or the famous Passenger for Apache and Nginx, or even some newcomers like Thin and Unicorn, every application has an ideal setup and its web server architecture greatly impacts its performance and memory usage.\\\\
The database choices, ranging from popular relational databases like MySQL to more recent projects like Cassandra or MongoDB are also associated with strong advantages and disadvantages. This aspect gains even greater importance when considering that the database is a major performance bottleneck.\\\\
Let's not forget about the application itself: a few coding conventions could be followed to improve the application's performance, scalability and, most importantly, the code's quality.\\\\
\textit{Your system, your architecture. It's all about choice.}

    \chapter{General Guidelines and Conventions for Optimizing Rails Applications} % (fold)
\label{ap:scaling_rails}

Building highly performant Ruby on Rails applications involves carefully examining all system components. The following sections will briefly analyze each component, exhibiting its alternatives and analyzing their benefits and shortcomings when compared to other possible solutions.

\section{Operating System}
From a performance perspective, Linux is the best operating system to use when deploying Rails applications. Comparing with other operating systems, MRI is:
\begin{itemize}
  \item approximately 100\% faster than its Windows counterpart;
  \item approximately 30\% faster than its BSD counterpart.
\end{itemize}
On a similar tone, YARV on Linux is:
\begin{itemize}
  \item approximately 70\% faster than its Windows counterpart;
  \item approximately 22\% faster than its BSD counterpart.
\end{itemize}
MySQL also yields better performance in UNIX systems, since Windows restrains its concurrent capabilities (by limiting the allowed number of opened files).\\\\
Gentoo is the most stable and scalable Linux distribution. It was able to scale up to 10000 concurrent requests on a simple page responding on an acceptable time span while Debian, Ubuntu and CentOS were not.\\\\
Regarding server configuration the kernel should be using the Deadline I/O scheduler, a non-preemptive configuration and a timer interrupt of 100Hz. Its sysctl configuration should also be changed to improve its scalability and stability, namely:
\begin{itemize}
\item net.core.rmem\_max --- 16777216
\item net.core.wmem\_max --- 16777216 
\item net.ipv4.tcp\_rmem --- 4096~\, 87380~\, 16777216
\item net.ipv4.tcp\_wmem --- 4096~\, 87380~\, 16777216
\item net.core.netdev\_max\_backlog --- 4096
\item net.core.somaxconn --- 4096
\item net.ipv4.tcp\_tw\_reuse --- 1
\item net.ipv4.tcp\_tw\_recycle --- 1
\item net.ipv4.tcp\_fin\_timeout --- 15
\item net.ipv4.tcp\_timestamps --- 0
\item net.ipv4.tcp\_orphan\_retries --- 1
\end{itemize}

\section{Ruby}
YARV is the Ruby interpreter with the best overall performance. Applications still using Ruby 1.8 can be easily ported to the new version and this upgrade is highly advisable. However, for those applications that have specific Ruby 1.8-related requirements, the best performing interpreter is JRuby.\\\\
Regarding YARV, there is a fork on GitHub which enables parameter configuration for adaptive performance at \url{http://github.com/goncalossilva/ruby}. To run Ruby with customized configurations, create the following script:
\begin{lstlisting}[language=bash]
export RUBY_HEAP_SLOTS_INCREMENT=500000
export RUBY_HEAP_MIN_SLOTS=500000
export RUBY_HEAP_SLOTS_GROWTH_FACTOR=1.1
export RUBY_GC_MALLOC_LIMIT=40000000
export RUBY_HEAP_FREE_MIN=100000
ruby
\end{lstlisting}
The supported parameters are the previously exhibited. Each addresses a different functionality:
\begin{itemize}
  \item RUBY\_HEAP\_SLOTS\_INCREMENT --- Initial number of heap slots. It also represents the minimum number of slots, at all times (default: 10000);
  \item RUBY\_HEAP\_MIN\_SLOTS --- The number of new slots to allocate when all initial slots are used (default: 10000);
  \item RUBY\_HEAP\_SLOTS\_GROWTH\_FACTOR --- Next time Ruby needs new heap slots it will use a multiplicator, defined by this environment variable’s value (default: 1.8, meaning it will allocate 18000 new slots if default settings are in use);
  \item RUBY\_GC\_MALLOC\_LIMIT --- The number of C data structures that can be allocated before triggering the garbage collector. This one is very important since the default value makes the GC run when there are still empty heap slots because Rails allocates and deallocates a lot of data (default: 8000000);
  \item RUBY\_HEAP\_FREE\_MIN --- The number of free slots that should be present after GC finishes running. If there are fewer slots than those defined it will allocate new ones according to the value of RUBY\_HEAP\_SLOTS\_INCREMENT and the previously mentioned value of RUBY\_HEAP\_SLOTS\_GROWTH\_FACTOR (default: 4096).
\end{itemize}
Each application has its own optimal configuration. Twitter's and 37signals' settings are shown bellow:
\begin{lstlisting}[language=bash]
# 37signals
RUBY_HEAP_MIN_SLOTS=600000
RUBY_GC_MALLOC_LIMIT=59000000
RUBY_HEAP_FREE_MIN=100000

# Twitter
RUBY_HEAP_MIN_SLOTS=500000
RUBY_HEAP_SLOTS_INCREMENT=250000
RUBY_HEAP_SLOTS_GROWTH_FACTOR=1
RUBY_GC_MALLOC_LIMIT=50000000
\end{lstlisting}
Developers should benchmark their applications to find an optimal value combination for the best performance results.

\section{Web Servers}
Thin, Unicorn and Passenger have very similar performances. However, Thin uses slightly less memory and accomplishes similar results.\\\\
Thin is optimized for fast clients so it should be used in combination with a reverse proxy which buffers requests/replies for slow clients. Nginx is the optimal choice for its improved performance, stability and memory usage.\\\\
Thin as an option to enable threading. This, however, generally decreases this web server's performance, so the advice is to avoid using it. Nginx, on the other hand, has a powerful configuration option since it allows developers to specify the event model to use. One must use the model optimized for the running operating system which, for Linux, is \textit{epoll}.

\section{Databases}
MySQL is the best performing database among relation databases. Non-relational solutions, however, trade enhanced read/write speeds for higher disk usage. MongoDB seems to be the best performing non-relational database at this moment. It is up to the developer to decide which type of database to use, considering the mentioned benefits and shortcomings.\\\\
In any case, \textit{caching} is very important since it generally significantly improves the database performance.\\\\
If an application is using MySQL, alternate Ruby database libraries should be considered, namely \textit{mysql2} or \textit{mysqlplus}. Both perform better than the default one, providing significant improvements in database interactions.

\section{Ruby on Rails}
Rails 3 is remarkably faster than Rails 2 and it is nearing its first stable release. Upon commencing new projects, this version should be carefully considered.\\\\
Despite the version in use, there are some common Rails' features which can significantly help improving an applications' performance, namely:
\begin{itemize}
  \item Eager loading should be used whenever a record and any number of associations are being fetched from the database;
  \item Transactions should be used to wrap consecutive writes to the database, on either CREATE and UPDATE statements;
  \item Magic finders should be generally avoidable. Normal finds are generally very readable, so improving their readability will not likely be worthy given the performance penalty involved;
  \item Fetching large groups of records should be done using ``find\_each'' or ``find\_in\_batches'' and not using the regular ``find'' method.
\end{itemize}
When paired with Nginx, Rails can benefit from its \textit{X-Accel-Redirect} feature on file downloads by using the plugin found at \url{http://github.com/goncalossilva/x-accel-redirect}.\\\\
Finally, profiling is the most important part of fine-tunning. The current Ruby/Rails profiling tools integrate seamlessly and support many data visualization types, including an hierarchical HTML call stack very useful for non-automated analysis. All these tools should be used to spot the existing bottlenecks in the application itself.

    %\include{appendix_b}
  \end{singlespace}
\end{document}
