\chapter*{Definitions} % (fold)
\label{cha:definitions}
\headermark{Definitions}
The following definitions are in alphabetical order.\\\\

\begin{description}
  \item[Ahead-of-time Compilation] Compilation of intermediate code into a system-dependent binary.
  \item[Continuations] C-like gotos for Ruby.
  \item[Coupling] The degree to which each program module relies on each one of the other modules.
  \item[Endian] Order of individually addressable sub-units within a longer data word in external memory.
  \item[EventMachine] A library for Ruby, C++ and Java that provides event-driven I/O using the Reactor pattern.
  \item[Fork] A processes' act of creating a copy of itself.
  \item[Green Threads] Threads scheduled by the Virtual Machine, emulating multi-threaded environments.
  \item[Just-in-time Compilation] Also known as dynamic translation, just-in-time compilation is the act of converting code at runtime prior to executing it natively.
  \item[Mark and Sweep] The first garbage collection algorithm, consisting of two blocking phases: a mark phase and a sweep phase.
  \item[RDoc] The embedded documentation generator for the Ruby programming language.
  \item[World Wide Web] System of interlinked hypertext documents contained on the Internet.
\end{description}

% chapter definitions (end)
