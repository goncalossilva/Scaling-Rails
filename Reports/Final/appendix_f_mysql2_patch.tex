\chapter{Lazy Type Casting in mysql2} % (fold)
\label{ap:mysql2_patch}

A patch which enables lazy type casting of fields on of Ruby's \textit{MySQL} database libraries, \textit{mysql2}.

\begin{lstlisting}
diff --git a/ext/mysql2_ext.c b/ext/mysql2_ext.c
index 3f791d3..fadea7b 100644
--- a/ext/mysql2_ext.c
+++ b/ext/mysql2_ext.c
@@ -370,7 +370,7 @@ static VALUE nogvl_read_query_result(void *ptr)
   return res == 0 ? Qtrue : Qfalse;
 }
 
-/* mysql_store_result may (unlikely) read rows off the socket */
+/*  may (unlikely) read rows off the socket */
 static VALUE nogvl_store_result(void *ptr)
 {
   MYSQL * client = ptr;
@@ -521,92 +521,8 @@ static VALUE rb_mysql_result_fetch_row(int argc, VALUE * argv, VALUE self) {
       }
       rb_ary_store(wrapper->fields, i, field);
     }
-
+    VALUE val = INT2NUM(wrapper->lastRowProcessed+i);
     if (row[i]) {
-      VALUE val;
-      switch(fields[i].type) {
-        case MYSQL_TYPE_NULL:       // NULL-type field
-          val = Qnil;
-          break;
-        case MYSQL_TYPE_BIT:        // BIT field (MySQL 5.0.3 and up)
-          val = rb_str_new(row[i], fieldLengths[i]);
-          break;
-        case MYSQL_TYPE_TINY:       // TINYINT field
-        case MYSQL_TYPE_SHORT:      // SMALLINT field
-        case MYSQL_TYPE_LONG:       // INTEGER field
-        case MYSQL_TYPE_INT24:      // MEDIUMINT field
-        case MYSQL_TYPE_LONGLONG:   // BIGINT field
-        case MYSQL_TYPE_YEAR:       // YEAR field
-          val = rb_cstr2inum(row[i], 10);
-          break;
-        case MYSQL_TYPE_DECIMAL:    // DECIMAL or NUMERIC field
-        case MYSQL_TYPE_NEWDECIMAL: // Precision math DECIMAL or NUMERIC field (MySQL 5.0.3 and up)
-          val = rb_funcall(cBigDecimal, intern_new, 1, rb_str_new(row[i], fieldLengths[i]));
-          break;
-        case MYSQL_TYPE_FLOAT:      // FLOAT field
-        case MYSQL_TYPE_DOUBLE:     // DOUBLE or REAL field
-          val = rb_float_new(strtod(row[i], NULL));
-          break;
-        case MYSQL_TYPE_TIME: {     // TIME field
-          int hour, min, sec, tokens;
-          tokens = sscanf(row[i], "%2d:%2d:%2d", &hour, &min, &sec);
-          val = rb_funcall(rb_cTime, intern_local, 6, INT2NUM(0), INT2NUM(1), INT2NUM(1), INT2NUM(hour), INT2NUM(min), INT2NUM(sec));
-          break;
-        }
-        case MYSQL_TYPE_TIMESTAMP:  // TIMESTAMP field
-        case MYSQL_TYPE_DATETIME: { // DATETIME field
-          int year, month, day, hour, min, sec, tokens;
-          tokens = sscanf(row[i], "%4d-%2d-%2d %2d:%2d:%2d", &year, &month, &day, &hour, &min, &sec);
-          if (year+month+day+hour+min+sec == 0) {
-            val = Qnil;
-          } else {
-            if (month < 1 || day < 1) {
-              rb_raise(cMysql2Error, "Invalid date: %s", row[i]);
-              val = Qnil;
-            } else {
-              val = rb_funcall(rb_cTime, intern_local, 6, INT2NUM(year), INT2NUM(month), INT2NUM(day), INT2NUM(hour), INT2NUM(min), INT2NUM(sec));
-            }
-          }
-          break;
-        }
-        case MYSQL_TYPE_DATE:       // DATE field
-        case MYSQL_TYPE_NEWDATE: {  // Newer const used > 5.0
-          int year, month, day, tokens;
-          tokens = sscanf(row[i], "%4d-%2d-%2d", &year, &month, &day);
-          if (year+month+day == 0) {
-            val = Qnil;
-          } else {
-            if (month < 1 || day < 1) {
-              rb_raise(cMysql2Error, "Invalid date: %s", row[i]);
-              val = Qnil;
-            } else {
-              val = rb_funcall(cDate, intern_new, 3, INT2NUM(year), INT2NUM(month), INT2NUM(day));
-            }
-          }
-          break;
-        }
-        case MYSQL_TYPE_TINY_BLOB:
-        case MYSQL_TYPE_MEDIUM_BLOB:
-        case MYSQL_TYPE_LONG_BLOB:
-        case MYSQL_TYPE_BLOB:
-        case MYSQL_TYPE_VAR_STRING:
-        case MYSQL_TYPE_VARCHAR:
-        case MYSQL_TYPE_STRING:     // CHAR or BINARY field
-        case MYSQL_TYPE_SET:        // SET field
-        case MYSQL_TYPE_ENUM:       // ENUM field
-        case MYSQL_TYPE_GEOMETRY:   // Spatial fielda
-        default:
-          val = rb_str_new(row[i], fieldLengths[i]);
-#ifdef HAVE_RUBY_ENCODING_H
-          // rudimentary check for binary content
-          if ((fields[i].flags & BINARY_FLAG) || fields[i].charsetnr == 63) {
-            rb_enc_associate_index(val, binaryEncoding);
-          } else {
-            rb_enc_associate_index(val, utf8Encoding);
-          }
-#endif
-          break;
-      }
       rb_hash_aset(rowHash, field, val);
     } else {
       rb_hash_aset(rowHash, field, Qnil);
@@ -651,20 +567,20 @@ static VALUE rb_mysql_result_each(int argc, VALUE * argv, VALUE self) {
         wrapper->lastRowProcessed++;
       }
 
-      if (row == Qnil) {
+      /*if (row == Qnil) {
         // we don't need the mysql C dataset around anymore, peace it
         rb_mysql_result_free_result(wrapper);
         return Qnil;
-      }
+      }*/
 
       if (block != Qnil) {
         rb_yield(row);
       }
     }
-    if (wrapper->lastRowProcessed == wrapper->numberOfRows) {
+    /*if (wrapper->lastRowProcessed == wrapper->numberOfRows) {
       // we don't need the mysql C dataset around anymore, peace it
       rb_mysql_result_free_result(wrapper);
-    }
+    }*/
   }
 
   return wrapper->rows;
@@ -686,6 +602,118 @@ static VALUE rb_raise_mysql2_error(MYSQL *client) {
   return Qnil;
 }
 
+static VALUE rb_mysql_result_cast(VALUE self, VALUE index) {
+  mysql2_result_wrapper * wrapper;
+  MYSQL_FIELD * field = NULL;
+  MYSQL_ROW row;
+  VALUE val;
+  unsigned long * fieldLengths;
+  void * ptr;
+  
+  GetMysql2Result(self, wrapper);
+  
+  if (wrapper->numberOfFields == 0) {
+    wrapper->numberOfFields = mysql_num_fields(wrapper->result);
+    wrapper->fields = rb_ary_new2(wrapper->numberOfFields);
+  }
+  
+  unsigned long r = index / wrapper->numberOfRows;
+  int c = index % wrapper->numberOfFields;
+  
+  ptr = wrapper->result;
+  mysql_data_seek(ptr, r); // <--- Segmentation fault
+  row = (MYSQL_ROW)rb_thread_blocking_region(nogvl_fetch_row, ptr, RUBY_UBF_IO, 0);
+  
+  field = mysql_fetch_field_direct(ptr, c);
+  fieldLengths = mysql_fetch_lengths(wrapper->result);
+  
+  switch(field->type) {
+    case MYSQL_TYPE_NULL:       // NULL-type field
+      val = Qnil;
+      break;
+    case MYSQL_TYPE_BIT:        // BIT field (MySQL 5.0.3 and up)
+      val = rb_str_new(row[c], fieldLengths[c]);
+      break;
+    case MYSQL_TYPE_TINY:       // TINYINT field
+    case MYSQL_TYPE_SHORT:      // SMALLINT field
+    case MYSQL_TYPE_LONG:       // INTEGER field
+    case MYSQL_TYPE_INT24:      // MEDIUMINT field
+    case MYSQL_TYPE_LONGLONG:   // BIGINT field
+    case MYSQL_TYPE_YEAR:       // YEAR field
+      val = rb_cstr2inum(row[c], 10);
+      break;
+    case MYSQL_TYPE_DECIMAL:    // DECIMAL or NUMERIC field
+    case MYSQL_TYPE_NEWDECIMAL: // Precision math DECIMAL or NUMERIC field (MySQL 5.0.3 and up)
+      val = rb_funcall(cBigDecimal, intern_new, 1, rb_str_new(row[c], fieldLengths[c]));
+      break;
+    case MYSQL_TYPE_FLOAT:      // FLOAT field
+    case MYSQL_TYPE_DOUBLE:     // DOUBLE or REAL field
+      val = rb_float_new(strtod(row[c], NULL));
+      break;
+    case MYSQL_TYPE_TIME: {     // TIME field
+      int hour, min, sec, tokens;
+      tokens = sscanf(row[c], "%2d:%2d:%2d", &hour, &min, &sec);
+      val = rb_funcall(rb_cTime, intern_local, 6, INT2NUM(0), INT2NUM(1), INT2NUM(1), INT2NUM(hour), INT2NUM(min), INT2NUM(sec));
+      break;
+    }
+    case MYSQL_TYPE_TIMESTAMP:  // TIMESTAMP field
+    case MYSQL_TYPE_DATETIME: { // DATETIME field
+      int year, month, day, hour, min, sec, tokens;
+      tokens = sscanf(row[c], "%4d-%2d-%2d %2d:%2d:%2d", &year, &month, &day, &hour, &min, &sec);
+      if (year+month+day+hour+min+sec == 0) {
+        val = Qnil;
+      } else {
+        if (month < 1 || day < 1) {
+          rb_raise(cMysql2Error, "Invalid date: %s", row[c]);
+          val = Qnil;
+        } else {
+          val = rb_funcall(rb_cTime, intern_local, 6, INT2NUM(year), INT2NUM(month), INT2NUM(day), INT2NUM(hour), INT2NUM(min), INT2NUM(sec));
+        }
+      }
+      break;
+    }
+    case MYSQL_TYPE_DATE:       // DATE field
+    case MYSQL_TYPE_NEWDATE: {  // Newer const used > 5.0
+      int year, month, day, tokens;
+      tokens = sscanf(row[c], "%4d-%2d-%2d", &year, &month, &day);
+      if (year+month+day == 0) {
+        val = Qnil;
+      } else {
+        if (month < 1 || day < 1) {
+          rb_raise(cMysql2Error, "Invalid date: %s", row[c]);
+          val = Qnil;
+        } else {
+          val = rb_funcall(cDate, intern_new, 3, INT2NUM(year), INT2NUM(month), INT2NUM(day));
+        }
+      }
+      break;
+    }
+    case MYSQL_TYPE_TINY_BLOB:
+    case MYSQL_TYPE_MEDIUM_BLOB:
+    case MYSQL_TYPE_LONG_BLOB:
+    case MYSQL_TYPE_BLOB:
+    case MYSQL_TYPE_VAR_STRING:
+    case MYSQL_TYPE_VARCHAR:
+    case MYSQL_TYPE_STRING:     // CHAR or BINARY field
+    case MYSQL_TYPE_SET:        // SET field
+    case MYSQL_TYPE_ENUM:       // ENUM field
+    case MYSQL_TYPE_GEOMETRY:   // Spatial fielda
+    default:
+      val = rb_str_new(row[c], fieldLengths[c]);
+#ifdef HAVE_RUBY_ENCODING_H
+      // rudimentary check for binary content
+      if ((field->flags & BINARY_FLAG) || field->charsetnr == 63) {
+        rb_enc_associate_index(val, binaryEncoding);
+      } else {
+        rb_enc_associate_index(val, utf8Encoding);
+      }
+#endif
+      break;
+  }
+  
+  return val;
+}
+
 /* Ruby Extension initializer */
 void Init_mysql2_ext() {
   rb_require("date");
@@ -709,6 +737,7 @@ void Init_mysql2_ext() {
   rb_define_method(cMysql2Client, "async_result", rb_mysql_client_async_result, 0);
   rb_define_method(cMysql2Client, "last_id", rb_mysql_client_last_id, 0);
   rb_define_method(cMysql2Client, "affected_rows", rb_mysql_client_affected_rows, 0);
+  rb_define_method(cMysql2Client, "cast", rb_mysql_result_cast, 1);
 
   cMysql2Error = rb_define_class_under(mMysql2, "Error", rb_eStandardError);
   rb_define_method(cMysql2Error, "error_number", rb_mysql_error_error_number, 0);
@@ -716,6 +745,9 @@ void Init_mysql2_ext() {
 
   cMysql2Result = rb_define_class_under(mMysql2, "Result", rb_cObject);
   rb_define_method(cMysql2Result, "each", rb_mysql_result_each, -1);
+  
+  //cMysql2Type = rb_define_class_under(mMysql2, "Type", rb_cObject);
+  //rb_define_singleton_method(cMysql2Type, "cast", rb_mysql_result_cast, 1);
 
   VALUE mEnumerable = rb_const_get(rb_cObject, rb_intern("Enumerable"));
   rb_include_module(cMysql2Result, mEnumerable);
diff --git a/lib/active_record/connection_adapters/mysql2_adapter.rb b/lib/active_record/connection_adapters/mysql2_adapter.rb
index 825dd44..7daa537 100644
--- a/lib/active_record/connection_adapters/mysql2_adapter.rb
+++ b/lib/active_record/connection_adapters/mysql2_adapter.rb
@@ -14,6 +14,12 @@ module ActiveRecord
   module ConnectionAdapters
     class Mysql2Column < Column
       BOOL = "tinyint(1)".freeze
+      
+      def initialize(name, default, connection, sql_type = nil, null = true)
+        @connection = connection
+        super(name, default, sql_type, null)
+      end
+      
       def extract_default(default)
         if sql_type =~ /blob/i || type == :text
           if default.blank?
@@ -48,43 +54,47 @@ module ActiveRecord
           when :boolean       then Object
         end
       end
-
+      
+      # Gets the value (from the index)
       def type_cast(value)
         return nil if value.nil?
         case type
-          when :string                then value
-          when :text                  then value
-          when :integer               then value.to_i rescue value ? 1 : 0
-          when :float                 then value.to_f # returns self if it's already a Float
-          when :decimal               then self.class.value_to_decimal(value)
-          when :datetime, :timestamp  then value.class == Time ? value : self.class.string_to_time(value)
-          when :time                  then value.class == Time ? value : self.class.string_to_dummy_time(value)
-          when :date                  then value.class == Date ? value : self.class.string_to_date(value)
-          when :binary                then value
-          when :boolean               then self.class.value_to_boolean(value)
-          else value
+          when :string    then Mysql2::Type.cast(value)
+          when :text      then Mysql2::Type.cast(value)
+          when :integer   then Mysql2::Type.cast(value).to_i rescue Mysql2::Type.cast(value) ? 1 : 0
+          when :float     then Mysql2::Type.cast(value).to_f
+          when :decimal   then self.class.value_to_decimal(Mysql2::Type.cast(value))
+          when :datetime  then self.class.string_to_time(Mysql2::Type.cast(value))
+          when :timestamp then self.class.string_to_time(Mysql2::Type.cast(value))
+          when :time      then self.class.string_to_dummy_time(Mysql2::Type.cast(value))
+          when :date      then self.class.string_to_date(Mysql2::Type.cast(value))
+          when :binary    then self.class.binary_to_string(Mysql2::Type.cast(value))
+          when :boolean   then self.class.value_to_boolean(Mysql2::Type.cast(value))
+          else Mysql2::Type.cast(value)
         end
       end
 
       def type_cast_code(var_name)
         case type
-          when :string                then nil
-          when :text                  then nil
-          when :integer               then "#{var_name}.to_i rescue #{var_name} ? 1 : 0"
-          when :float                 then "#{var_name}.to_f"
-          when :decimal               then "#{self.class.name}.value_to_decimal(#{var_name})"
-          when :datetime, :timestamp  then "#{var_name}.class == Time ? #{var_name} : #{self.class.name}.string_to_time(#{var_name})"
-          when :time                  then "#{var_name}.class == Time ? #{var_name} : #{self.class.name}.string_to_dummy_time(#{var_name})"
-          when :date                  then "#{var_name}.class == Date ? #{var_name} : #{self.class.name}.string_to_date(#{var_name})"
-          when :binary                then nil
-          when :boolean               then "#{self.class.name}.value_to_boolean(#{var_name})"
+          when :string    then "Mysql2::Type.cast(#{var_name})"
+          when :text      then "Mysql2::Type.cast(#{var_name})"
+          when :integer   then "(Mysql2::Type.cast(#{var_name}).to_i rescue Mysql2::Type.cast(#{var_name}) ? 1 : 0)"
+          when :float     then "Mysql2::Type.cast(#{var_name}).to_f"
+          when :decimal   then "#{self.class.name}.value_to_decimal(Mysql2::Type.cast(#{var_name}))"
+          when :datetime  then "#{self.class.name}.string_to_time(Mysql2::Type.cast(#{var_name}))"
+          when :timestamp then "#{self.class.name}.string_to_time(Mysql2::Type.cast(#{var_name}))"
+          when :time      then "#{self.class.name}.string_to_dummy_time(Mysql2::Type.cast(#{var_name}))"
+          when :date      then "#{self.class.name}.string_to_date(Mysql2::Type.cast(#{var_name}))"
+          when :binary    then "#{self.class.name}.binary_to_string(Mysql2::Type.cast(#{var_name}))"
+          when :boolean   then "#{self.class.name}.value_to_boolean(Mysql2::Type.cast(#{var_name}))"
           else nil
         end
       end
 
       private
         def simplified_type(field_type)
-          return :boolean if Mysql2Adapter.emulate_booleans && field_type.downcase.index(BOOL)
+          puts field_type
+          return :boolean if Mysql2Adapter.emulate_booleans && @connection.cast(field_type).downcase.index(BOOL)
           return :string  if field_type =~ /enum/i
           return :integer if field_type =~ /year/i
           super
@@ -414,7 +424,7 @@ module ActiveRecord
         columns = []
         result = execute(sql, :skip_logging)
         result.each(:symbolize_keys => true) { |field|
-          columns << Mysql2Column.new(field[:Field], field[:Default], field[:Type], field[:Null] == "YES")
+          columns << Mysql2Column.new(field[:Field], field[:Default], @connection, field[:Type], field[:Null] == "YES")
         }
         columns
       end
@@ -592,4 +602,4 @@ module ActiveRecord
         end
     end
   end
-end
\ No newline at end of file
+end
\end{lstlisting}
