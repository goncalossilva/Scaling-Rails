\chapter{Introduction} % (fold)
\label{cha:introduction}
\headermark{Introduction}

\section*{} % (fold)
This chapter briefly presents the project's context, purpose and scope. It covers the motivation behind it and its objectives.% while also detailing the report structure, providing an overview of each of the remaining chapters.

% section  (end)

\section{Context} % (fold)
\label{sec:context}
The internet started in the early 1990s as a work tool for CERN. It evolved into a vast information repository for the public~\cite{teaching_webdev_web20}---the Web---and had 16 million users back in 1995. The year after Netscape went public and achieved an impressive 90\% market share. It quickly lost its prominence during the first browser war, conceding its leadership to Microsoft's Internet Explorer~\cite{browser_wars}. The Web was starting to become a presence in everyday life. In 2009, 14 years later, the Web had more than 1700 million users and its popularity keeps growing nowadays~\cite{internet_stats}.

The Web is becoming an essential pillar of many businesses, social networks, gaming industries \textit{et al.} since distance is no longer an issue when it comes to information exchange.  It begins to present itself as a critical presence on computers nowadays. Many people believe future applications will mostly be web-based, pushing internet's importance even further~\cite{browser_application_platform}.

The growth of internet usage and its impact in corporate applications implies all kinds of particularities. First of all, users must trust the web. This is an essential pillar of any web service. To build customer trust, service providers must pay attention to their users' needs and desires and they must meet their expectations~\cite{trust_semantic_web}. 

In the fall of 2001, many people concluded that the web was overhyped and bloated. There was a need for richer content and better user experiences~\cite{oreilly_web20}. In this context, the term \textit{Web 2.0} was born. Its meaning is not well established but a commonly accepted definition states that it consists in the improvement of the first version of the Web~\cite{rubyonrails_tutorial} and involves many core concepts like usability and dynamic content~\cite{what_is_web20}.

Side by side with the Web's growth were the increasing user expectations when it comes to their experiences. As time goes by, users demand better interactions with the services they use. Their user experiences are partially based on response times, responsiveness and performance of Web applications~\cite{prioritizing_web_usability}. These concepts are part of the key factors to their success~\cite{responsiveness}. Users expect waiting times to be kept at an acceptable minimum and whenever they feel that this expectation is not being met their trust on the service diminishes. When users enter a given website they have a limited patience, related to their expectations and previous experiences. Whenever the system fails to meet those expectations, their patience decreases and it can cause them to leave. Steve Krug\footnote{Steve Krug is the author of the renowned book about human computer interaction and web usability entitled \textit{Don't Make Me Think}} entitles this phenomenon as the \textit{Reservoir of Goodwill}~\cite{dont_make_me_think}.

This increasing popularity, dependence and demand on the Web strives the developers for better tools to build quality web applications. Most developers seek the ability to increase their productivity while being able to build more complex, full-featured systems that suit their users' needs, either it is a social or business-oriented service~\cite{comparison_agile_frameworks}.

With the growing importance of Web applications, many tools emerged trying to make the developer's life easier. \textit{Web 2.0} improved the internet and created a new set of needs and expectations. This need motivated the development of powerful frameworks and, among many others, the Ruby on Rails framework\footnote{\url{http://rubyonrails.org}} was born~\cite{what_is_web20}. As the community puts it~\cite{rubyonrails}:
\begin{quote}
  ``Ruby on Rails$^{\scriptsize \textregistered}$~is an open-source web framework that's optimized for programmer happiness and sustainable productivity. It lets you write beautiful code by favoring convention over configuration.''
\end{quote}
Ruby on Rails is one of the best-known Ruby frameworks~\cite{agile_webdevelopment_with_rails}. Most people, from individuals to companies, want a \textit{Web 2.0} killer application and Rails seems an excellent way of achieving it~\cite{oo_business_models}. This framework is commonly associated with the \textit{Web 2.0} concept, along with AJAX~\cite{spaghetti_code}. It is also deeply related with \textit{Agile Web Development}~\cite{agile_webdevelopment_with_rails}. Rails allows developers to build high quality applications with smaller effort---less time, less lines of code and less files, always with low coupling~\cite{maintainability_web_applications_j2ee_dotnet_ror}.

As Tim O'Reilly\footnote{Tim O'Reilly is the founder of O'Reilly Media and a supporter of the free software and open source movements} states~\cite{oreilly_ror}:
\begin{quote}
  ``Ruby on Rails is a breakthrough in lowering the barriers of entry to programming. Powerful web applications that formerly might have taken weeks or months to develop can be produced in a matter of days.''
\end{quote}
The huge success achieved by this framework made it jump into a spotlight, with many companies starting to use it as the development framework for their applications. Some widely known services like \textit{Basecamp}\footnote{\url{http://basecamphq.com/}}, \textit{Twitter}\footnote{\url{http://twitter.com/}}, \textit{Hulu}\footnote{\url{http://hulu.com/}}, \textit{YellowPages}\footnote{\url{http://yellowpages.com/}} or \textit{GitHub}\footnote{\url{http://github.com/}} push this platform even further by giving it even more popularity~\cite{rubyonrails_applications}.
\begin{description}
  \item[Basecamp,] the original Rails application~\cite{rubyonrails_applications}, is an online project management tool which features live collaboration and task aiding software. It gathered over 1 million users since 2006~\cite{basecamp_turns_1000000}. It is ranked 516$^{th}$ on the \textit{Alexa Traffic Rank}~\cite{alexa}.
  \item[Twitter,] a real-time short messaging service that works on multiple networks and devices. It is used for quick sharing of information, either updates from friends or breaking world news updates. It had more than 18 million adult users in the USA by the end of 2009 and is expected to achieve an impressive quantity of 26 million adult users in the same country by 2010~\cite{emarketer_twitter_usage}. It is ranked 11$^{th}$ on the \textit{Alexa Traffic Rank}~\cite{alexa}.
  \item[Hulu,] a TV and Movie streaming website which allows users to watch their favorite shows on the browser for free. It had a significant amount of traffic in 2009, only staying behind Google services and \textit{Fox Interactive Media}~\cite{hulu_growth}. It ranks 177$^{th}$ on the \textit{Alexa Traffic Rank}~\cite{alexa}.
  \item[YellowPages,] a service that indexes and provides business listings of the United States of America, allowing users to search for services they are looking for, among other features. It ranks 929$^{th}$ on the \textit{Alexa Traffic Rank}. To note that it is a USA-oriented application, raking 161$^{th}$ if the scope is limited to the USA~\cite{alexa}.
  \item[GitHub,] one of the best know repository hosting system which works with the Git VCS. It currently has more than 185 thousand registered developers and although being tightly associated with public open-source projects, it also supports private development. It ranks 997$^{th}$ on the \textit{Alexa Traffic Rank}~\cite{alexa}.
\end{description}
These systems have high scalability demands. In order to keep increasing their popularity and to keep building users' trust, they also need to provide a great user experience which involve reasonable response times while serving thousands of requests concurrently.

Some press reports question Rails' ability to scale, mainly based on the issues \textit{Twitter} faced when its growth reached a given magnitude~\cite{interview_alex_payne}. However, most of the issues were demystified as a software architecture design issue, taking the blame off of Ruby on Rails~\cite{ror_ecosystem_whitepaper}. Nonetheless, despite all the advantages this framework possesses, scalability is not one of them. Although not being as scalable as PHP or Java, Ruby offers higher development speeds~\cite{issues_web_frameworks}.

Luckily, only high traffic websites have to get deeply involved in scalability details. However, developers should always aim at building highly performant applications from the start. They should be able to build Rails-based high-quality applications whose scalability is not directly related to hardware upgrades~\cite{interview_alex_payne}. Issues should be identified and solutions proposed so that this acclaimed Ruby framework becomes more scalable \textit{out of the box}, diminishing its dependence on hardware upgrades or major architectural changes. Developers should be aware of their choices' benefits and shortcomings. This way, Ruby on Rails development happiness can not only last through the creation of a web platform, but also through its maintenance.
% section context (end)

\section{Motivation and Objectives} % (fold)
\label{sec:motivation_and_objectives}
The Web starts to play a critically important role in many people's lives, either from a professional or personal point of view. User experiences have become of great importance in recent times, with \textit{Web 2.0} raising the expectations on better interactions.

Internet accesses keep increasing in number and the recent developments in the smartphone, tablet and netbook's areas help inflating this growth rate even further. More people are starting to be permanently connected to the internet~\cite{npd:3g,mobileweb,netbooks}.

Users expect the Web to work as they preconceived and this fact has a great focus on recent developments in Interaction Design~\cite{interaction_design}. As innovation pushes user experiences to a new level, with richer information presented and organized in ways never seen before, technologies tend to emerge to support such evolutionary content and forms of organization. Developers need to meet the users' needs and they do not have unlimited time to do it, thus many recent frameworks have gained notorious popularity for being agile and robust, increasing productivity rates to a higher level~\cite{trends_webdev}. Ruby on Rails, as most recent frameworks, offers convenient methods and features which greatly improve the product's quality without the need for extended development times. However, it also makes it harder for the development team to build a highly scalable application when there are limited hardware resources~\cite{look_common_performance_problems_rails}.

Scaling and performance optimization should not be so hard to achieve in Rails, though. Many \textit{Web 2.0} platforms are created everyday and Rails-related scalability issues should not be an obstacle to their success. The framework's purpose is to help developing high-quality applications, not limiting their accomplishments to a given number of concurrent users. The Web should be able to shine in all of its glory and tools like Ruby on Rails exist to allow it to improve and innovate further and further, meeting the users' increasing standards and demands.

Performance optimization has been a work focus since computers were born~\cite{mass_memory_system_optimization}. Many people have focused on optimizing many different components, like the Ruby interpreter~\cite{yarv}, Rails itself~\cite{rails_merb_merge_performance} or the superjacent application~\cite{scaling_rails_bottomup,vaporware_to_awesome,rebuilding_scaling_yellowpages,5tips_scale_ror}. Many have focused on improving the speed and scalability of databases~\cite{performance_analysis_db_arch} and webservers~\cite{webserver_scheduling}. Others look for potential issues in the Operating System and end up patching and tweaking the system's configuration~\cite{unix_os_comparison, architecture_impact_os}. One can infer that most of the performance optimization activities focus on single elements or try to find out a single culprit to blame. It is also necessary to envision performance optimization as a holistic activity. If a small part of the system changes it will affect all those who interact with it by smaller or larger margins. In Chaos Theory this is called \textit{Sensitive Dependence} or, as more commonly known, the \textit{Butterfly Effect}. As mentioned in \textit{Quantum Chaotic Environments}~\cite{butterfly_effect_quote}:
\begin{quote}
  ``The exponential divergence (\ldots) from slightly different initial conditions---the famous butterfly effect---is a fingerprint of chaos in classical mechanics.''
\end{quote}
Rails is a highly dependent system. There are many components involved and all of them can be optimized. The key concept is that in order to improve a Rails application's scalability, the task should be addressed with a deep notion of its associated components and the available alternatives. The whole system performance is what truly matters, not the performance of its individual parts. The core mindset of this project is to address all involved components from Rails' perspective.

There is the need for a solid set of general development conventions and guidelines oriented towards the scalability and performance of Rails projects. Developers seek optimal configurations for all the components involved so they can wring every processing cycle out of their applications, in order to increase their scalability and decrease their response times. There is urgency in looking at all components Rails depends on, determining which are best for which situation and tune them to suit Rails' needs---the system, envisioned as a whole.

Improving user experiences while profiling applications is also imperative. Profiling is a critical activity when optimizing applications but recent versions of Ruby and Rails break these tools and older versions only supported text-based outputs. Profiling should be functional, less verbose and more intuitive.

Finally, increasing the global awareness of the importance of building highly performant applications is also important. Developers should consider this aspect from the beginning and have access to all the information they need to plan, architecture and develop scalable systems.

These activities will be mostly applied in \textit{Escolinhas.pt}\footnote{\url{http://escolinhas.pt}}, a rapidly growing Portuguese Rails-based project. \textit{Escolinhas.pt} aims at sustaining social and collaborative work for children in elementary schools involving students, teachers and parents as its users. With the user demand increasing day by day, it becomes an excellent case-study application to research, test and apply most of the work and discoveries made during the course of this thesis.

Producing the aforementioned generic conventions and guidelines, fixing and greatly improving Rails-related profiling tools and increasing this subjects' awareness inside the Ruby on Rails community are the main goals of ``Scaling Rails: a system-wide approach to performance optimization''.
% section motivation_and_objectives (end)


\section{Report Overview} % (fold)
\label{sec:report_overview}
The rest of this report is structured as follows.
\begin{description}
  \item[Chapter~\ref{cha:technologies_review}: ``Technologies Review''] gives an overview over the involved technologies mentioned in this research, providing important background information on each one of them.
  \item[Chapter~\ref{cha:state_of_the_art}: ``State of the Art''] reviews each component alternative's performance and scalability and provides other important details for the problem approach.
  \item[Chapter~\ref{cha:problem_statement}: ``Problem Statement''] contains the problem to be addressed and throughly explains it, also exposing its usefulness.
  \item[Chapter~\ref{cha:problem_approach_and_results}: ``Problem Approach and Results''] benchmarks each component's alternatives, analyzes their settings and possible configurations, and details the developments made concerning each component.
  \item[Chapter~\ref{cha:conclusions}: ``Conclusions''] reviews the project, drawing conclusions about the issues addressed in ~\ref{cha:problem_statement}. It also provides a summary of contributions and some insights on which future developments have been considered.
\end{description}

% chapter introduction (end)
