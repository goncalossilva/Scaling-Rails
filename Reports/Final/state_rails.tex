\section{Ruby on Rails} % (fold)
\label{state:sec:ruby_on_rails}
Rails is developed by a huge community and coordinated by its core team~\cite{rails_core_team}. However, many companies like \textit{Twitter} had the need to squeeze this frameworks' scalability and focused on some identified critical bottlenecks like Ruby's garbage collector mentioned in section~\ref{state:sec:ruby}.

There have been some outside performance improvements over Rails 2.3 code, though. From simple Rails' source minor patching~\cite{accunote_rails} to complete architecture makeovers~\cite{distributed_rails}, Ruby on Rails' improvement attempts were widely attempted and some succeeded.

However, the main performance improvements targeted at Rails 2.3 can be found in Rails 3. The decoupling mentioned in section~\ref{tech:sec:ruby_on_rails:rails3} significantly improved Rails performance and scalability, allowing faster execution times and lighter memory usage.

There were also specific performance optimizations, namely on \textit{partial} and \textit{collection} handling. Initial performance benchmarks show that partial and collection rendering and overall performance is more than two times faster~\cite{vaporware_to_awesome,rails_merb_merge_performance}. Controller-related code was also refactored to become lighter and more modular. So was \textit{ActiveRecord}, which now uses \textit{Arel}---a relational algebra framework---to generate smarter and more efficient queries before dispatching them to the database adapter.

